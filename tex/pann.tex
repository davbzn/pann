\documentclass[12pt,a4paper]{book}
\usepackage[utf8]{inputenc}
\usepackage[english]{babel}
\usepackage{amsmath}
\usepackage{amsfonts}
\usepackage{amssymb}
\usepackage{graphicx}
\usepackage{wrapfig}
\usepackage{subcaption}
\usepackage[left=2cm,right=2cm,top=2cm,bottom=2cm]{geometry}
\usepackage{siunitx}

% Nice captions.
\usepackage[format=hang,font=small,labelfont=bf]{caption}
\setlength{\captionmargin}{25pt}

% define the symbols := and =: for ease of use
\usepackage{mathtools}
\newcommand{\defeq}{\vcentcolon=}
\newcommand{\eqdef}{=\vcentcolon}
    
% define bra and ket
\newcommand{\bra}[1]{%
	\left\langle #1 \right|
}
\newcommand{\ket}[1]{%
	\left| #1 \right\rangle
}
\newcommand{\bracket}[2]{%
	\left\langle #1 | #2 \right\rangle
}

% multiline commenting \begin{comment}
\usepackage{comment}
% package to cancel obliquely symbols
\usepackage{cancel}

% generate correct 1-bold (identity operator) font
\usepackage{dsfont}
% generate correct bold greek letters
\usepackage{bm}
% to produce stroked integral \fint
\usepackage{esint}

% better boxes (also multiple equations)
\usepackage{tcolorbox}
\tcbuselibrary{theorems}

% author managing package
\usepackage{authblk}

\title{Photonic Artificial Neural Networks}
\author[]{Davide Bazzanella}
\affil[]{Department of Physics, University of Trento}

% to create an index with terms and the corresponding page(s)
\usepackage{makeidx}
\makeindex

% create plots
\usepackage{pgfplots} % README here http://pgfplots.sourceforge.net/pgfplots.pdf
\pgfplotsset{compat=1.14} 	%ShareLaTeX wants \pgfplotsset{compat=1.14}

% externalize tikz images
\usepackage{tikz}
\usetikzlibrary{external}
\tikzexternalize[prefix=tikz/] % activate!

% draws bloch sphere
%\usepackage{blochsphere}

\begin{document}

% need this to fill between functions
\usepgfplotslibrary{fillbetween}
\usetikzlibrary{patterns,decorations,arrows, shapes.geometric}
\usetikzlibrary{arrows.meta}

% define the box around functions
\newtcolorbox{mymathbox}[1][]{colback=white, sharp corners, #1}

\maketitle
\tableofcontents
%\cleardoublepage
\clearpage

\chapter{Introduction}
WHAT and WHY?
BACKUP project, why silicon photonics (CMOS compatibility)

\chapter{Artificial Neural Networks}
THEORY ON ANNs
Artificial Neural Networks (ANNs) are computing devices which operate in way that mimics biological neural networks.
\section{Types of ANNs}
\subsection{Feed Forward NN}
\subsubsection{FF NN node}
%\begin{wrapfigure}[5]{r}{0.4\textwidth}
\begin{figure}[ht]
	\centering
	\begin{tikzpicture}
		\node [shape=circle, minimum size=2cm, draw=gray!100, fill=gray!20]%
					(af) at (2,0) {};
		\node at (af.south) [below] {activation function};
		\node [regular polygon, regular polygon sides=6, minimum size=2cm, draw=gray!100, fill=gray!20]%
					(ws) at (-2,0) {};
		\node at (ws.south) [below] {weighted sum};
		\draw (ws) to (af);
		\foreach \i in {0,...,3,5}
		\draw (-4.5,1.5-0.6*\i) .. controls (-3.5,1.5-0.6*\i) .. (ws);
		\draw [dashed] (-4.5,1.5-0.6*4) .. controls (-3.5,1.5-0.6*4) .. (ws);
%		\node at (af) [circle, radius=1cm, draw=gray!100, fill=gray!20] {};
%		\node (c) at (90:1) {$\textbf{r}_0$};
%		\draw [gray!50] (e) to (a);
%		\draw (g) [->] to (c);
%		\node at (0,0) [circle,draw=gray!100,fill=gray!20] {};
%		\draw (0.5,-0.5) [<->] to (0.5,0.5);
	\end{tikzpicture}
	\caption{Feed Forward node}
	\label{fig:FF_node}
\end{figure}
%\end{wrapfigure}
\subsection{Other Types of ANNs}

\chapter{Photonics applied to ANNs}
\section{Weighted sum of inputs}
\section{Nonlinear activation function}

\chapter{Device, setup and measurements}
\end{document}