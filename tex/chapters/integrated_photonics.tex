\chapter{Integrated Photonics}
\label{ch:Integrated_Photonics}

%\section{Introduction}
%\label{sec:Photonics_intro}
To build an artificial neuromorphic network one has to choose first some physical phenomenon to employ in the fundamental blocks, likewise transistors in electronic circuits use the various behaviors of electrons in resistances, capacitors, and inductances.
The physics which I want to build my artificial neuromorphic network with is photonics, precisely integrated silicon photonics.

Photonics is the physical science which studies detection, manipulation, and emission of light.
Specificially, integrated photonics is the branch that studies how to reduce and \textit{integrate} once macroscopic optical devices in miniaturized structures.
In the past few decades, thanks to the constant improvement of the manufacturing techniques, many productive problems has been progressively resolved.
Moreover interest in the field is rising, driven by the growing needs of communication technology, which was in turn following the increase in computational power of electronics.
Many integrated devices have been proposed \cite{??} and some of them even commercialized \cite{??} .

Silicon photonics is the research field that studies how to integrate optical devices in structures built with silicon and materials derived from it.
Silicon is well-known material which has been widely studied in microelectronics.
It possesses many qualities which allow relatively easy manufacturing of high grade structures.
However there also some drawbacks, for example being a centrosymmetric element it does not possess nonlinearities of the second order.

% Many other devices such as Mach-Zenhder interferometers (MZI) and Directional Couplers (DC) have been demonstrated.
% , Arrayed Waveguide Gratings (AWG) and Multimode interference(MMI)
%\subsection{History}

%\subsection{Silicon Photonics}
%\label{ssec:Silicon_Photonics}
\section{Silicon Photonics}
\label{sec:Silicon_Photonics}
Silicon-on-Insulator (SOI) photonics is one of the widest branches of integrated photonics.
It pursues the same objectives of all integrated photonics with the addition of following the production steps already developed for microelectronics.

Its framework is the silicon wafer, which is then altered by a often numerous series of production steps.
The first step is the burial of an insulator layer of $SiO_2$, called \textit{buffer}, which defines the bottom surface of the layer at the top.
The pure $Si$ volume at the top, the \textit{device} layer, is where integrated optical structures will be built with various processes like photolitography, thermal oxidation, ion implantation, and etching.

\begin{figure}[ht]
	\centering
	\tikzsetexternalprefix{tikz/}	% set subfolder
\tikzsetnextfilename{SOI}
\begin{tikzpicture}[baseline]
	\fill [pattern=north west lines, pattern color=gray!50]
				(0,0) rectangle (7,2);
	\fill [pattern=north east lines, pattern color=gray!50!red]
				(0,2) rectangle (7,3.5);
	\fill [pattern=north west lines, pattern color=gray!50]
				(0,3.5) rectangle (7,4);
	\draw [draw, color=gray!50] (0,0) rectangle (7,4);
	
	\fill [top color=blue!20, bottom color=white] (0,4) rectangle (7,6);
	
	
	\draw [top color=blue!20, bottom color=white] (7.5,5.5)
				++(0cm,-0.2cm) rectangle ++(0.6cm,0.3cm)
				++(0cm,-0.15cm) node [right] {Air};
	\draw [pattern=north west lines, pattern color=gray!50] (7.5,5.5)++(0,-0.6)
				++(0cm,-0.2cm) rectangle ++(0.6cm,0.3cm)
				++(0cm,-0.15cm) node [right] {$SiO$};
	\draw [pattern=north east lines, pattern color=gray!50!red] (7.5,5.5)++(0,-1.2)
				++(0cm,-0.2cm) rectangle ++(0.6cm,0.3cm)
				++(0cm,-0.2cm) node [right] {$SiO_2$};%++(0,-0.2) 
\end{tikzpicture}
	\caption{SOI productive framework.
		The buffer layer of $SiO_2$ is buried underneath the pure crystalline $Si$ device layer which is used to build structures.
		The top silicon layer will become the volume in which light will be confined.
		The bottom silicon, underneath the buffer layer, is called substrate.
		}
	\label{fig:SOI}
\end{figure}

The fact of relying on the SOI framework to build integrated devices is a strength of silicon photonics, because it enables researchers to develop CMOS compatible integrated optical structures.
Complementary Metal-Oxide-Semiconductor (CMOS) compatibility is an industry standard created to fabricate microelectronic devices.
Therefore by exploiting this standard all the know-how of the manufacturing industries behind the commercial microelectronic products is available to fabricate integrated photonics devices.

Moreover, by sharing the fabrication technology with microelectronic makes it easier to integrate electronic components to obtain hybrid integrated optoelectronic devices or to compensate lack of appropriate optical structures.

Hence silicon photonics is a promising technology for a combination of cost and technological reasons.
A list of advantages and disadvantages of silicon photonics in compared to other photonics technologies is provided in \autoref{tab:silicon_adv_and_disadv}.

\begin{table}[!htbp]
	\centering
	\footnotesize
	\begin{tabular}{p{0.45\textwidth} p{0.45\textwidth}}
	\toprule
	\normalsize Disadvantages & \normalsize Advantages \\ 
	\midrule
	\begin{enumerate}[ label={\roman*.}, noitemsep ]
		\item Stable, well-understood material
		\item Stable native oxide available for cladding and electrical isolation
		\item Relatively low-cost substrates
		\item Optically transparent at important wavelengths of \SI{1.3}{\um} and \SI{1.55}{\um}
		\item Well-characterized processing
		\item Highly confining optical technology
		\item High refractive index means short devices
		\item Micromachining means V-grooves and an effective hybrid technology are possible
		\item Semiconductor material offers the potential of optical and electronic integration
		\item High thermal conductivity means tolerance to high-power devices or to high packing density
		\item Carrier injection means optical modulation is possible
		\item Thermo-optic effect means a second possibility for optical modulation exists
	\end{enumerate}
	&
	\begin{enumerate}[ label={\roman*.}, noitemsep]
		\item No Pockels effect
		\item indirect bandgap means that native optical sources are not possible
		\item High refractive index means that inherently short devices which are difficult to fabricate (e.g. gratings)
		\item Modulation mechanisms tend to be relatively slow
		\item Thermal effects can be problematic for some optical circuits
	\end{enumerate}
	\\	
	\bottomrule
	\end{tabular}
	
	\caption{Advantages and disadvantages of silicon photonics over other integrated photonics technologies. Taken from \cite{Reed2004}.}
	\label{tab:silicon_adv_and_disadv}
\end{table}

\section{Guided-wave photonics}
\label{sec:guided-wave_photonics}
The most important thing for integrated photonics is the way light is confined and manipulated into microscopic structures.
While in conventional optics, light is delivered with bulky mirrors and lenses toward the desired position, in integrated photonics waveguides are the main device for this task.

\subsection{Waveguides}
\label{ssec:waveguides}
A waveguide is a path inside a medium, whose volume is defined by a certain number of interfaces between different materials, in which light remains confined and ideally travels with negligible losses.
This volume is called \textit{core} of the waveguide, while the medium external to it, if present, is called \textit{cladding.}

One can distinguish two types of waveguides: metallic and dielectric.
The former are based on the reflection of the electromagnetic field by the metallic surface.
Light is confined in the desired path by a series of metallic mirrors.
However, this mechanism works well only for electromagnetic radiation for which metals can still be considered \textit{perfect metals}.
For visible and infrared frequencies integrated metallic waveguides are not possible, due to too high absorption of the metals at optical frequencies.

On the other hand, dielectric waveguides are based on the phenomenon of total internal reflection (TIR) on the interface between two dielectric media.
To produce TIR, the two materials must have sufficiently different real part of the refractive index.
Moreover, they must also be transparent, i.e. they must have low imaginary part of the refractive index, in the required range of frequencies.
Hence not all media are suitable to build dielectric waveguides that work at a specific wavelength.
For example, silicon is known to be transparent for light at and around \SI{1.55}{\um}, among other wavelengths.
This feature together with a low distortion of signals is the reason why it is the most used wavelength in communication technology.

\subsubsection{Dielectric Waveguides}
\label{sssec:Dielectric_Waveguides}
Total internal reflection is the well-known phenomenon in which light coming from a material with high refractive index $n_H$ gets reflected at the interface with a material with low refractive index $n_L$.
For this to happen, incident light must be at an angle greater than the critical angle\index{critical angle} given by the Snell's law\index{Snell's law}:
\begin{equation*}
	\theta_C = \arcsin \left( \dfrac{n_L}{n_H}\right).
\end{equation*}

Waveguides have many shapes, however one can initially discriminate them by the dimensionality of the confinement of light.
The simplest one is called \textit{slab} and it is composed by two interfaces, which divides the material in three volumes, as shown in \autoref{fig:slabWG} below.
This type is classified as a 1D-waveguide, because it constrains light in only one dimension.

\begin{figure}[ht]
	\centering
%	\includegraphics[scale=.4]{figures/dielBC.pdf}
	\tikzsetexternalprefix{tikz/}	% set subfolder
\tikzsetnextfilename{dielWG}
\begin{tikzpicture}[baseline]
	\begin{axis}[	width=\textwidth*0.75,
								height=200pt,
								axis lines=center,
								xlabel={$z$}, ylabel={$y$},% zlabel={$y$},
								xmin=-0.1, xmax=+4.0,
								ymin=-1.6, ymax=+2.5,
								domain=0.5:3,
								xtick=\empty,
								ytick={-0.5,0.5},
								yticklabels={$-\dfrac{d}{2}$,$\dfrac{d}{2}$,},
								thick,
								area legend,
								legend style={draw=none},
								legend cell align=left,
								legend pos=outer north east,
								]

		\addplot [name path = top,] {+1.50};
		\addlegendentry{$n_L$}
		\addplot [name path = upper, 	forget plot] {+0.50};
		\addplot [name path = lower, 	forget plot] {-0.50};
		
		\addplot [pattern=crosshatch dots, pattern color=gray!50]
			fill between [ of=lower and upper];
		\addlegendentry{$n_H$}
		
		\draw [black] (0.5,-1.5) -- (3,-1.5);
		\draw [black] (0.5,-1.5) -- (0.5,+1.5);
		\draw [black] (3.0,-1.5) -- (3.0,+1.5);
		\draw [black] (3.0,-1.5) -- (3.8,-0.9) -- (3.8,+2.1) -- (3,+1.5);
		\draw [black] (0.5,+1.5) -- (1.3,+2.1) -- (3.8,+2.1);
		
		\path [name path= lowerFace, black] (3,-0.5) -- (3.8,+0.1);
		\path [name path= upperFace, black] (3,+0.5) -- (3.8,+1.1);
		\addplot [pattern=crosshatch dots, pattern color=gray!50]
			fill between [ of=lowerFace and upperFace];
		\draw [black] (3,-0.5) -- (3.8,+0.1) -- (3.8,+1.1) -- (3,+0.5);
		
		\draw [black, dashed] (1.3,+1.1) -- (3.8,+1.1);
		\draw [black, dashed] (1.3,+1.1) -- (1.3,+2.1);
		\draw [black, dashed] (1.3,+1.1) -- (0.5,+0.5);
		
		\draw [-stealth] (0,0) -- (0.4,0.3) node [above left] {$x$};
		
		\draw [	decoration={ markings, mark=between positions 0.0625 and 1 step 0.125 with {\arrow[line width=1]{stealth} }, }, postaction={decorate}]
			(0.8,-0.0) -- (0.88,+0.5) -- (1.06,-0.5) node [below] {\footnotesize Unguided ray} -- (1.22,+0.5) -- (1.38,-0.5) -- (1.46,+0.0);
			
		\draw [	decoration={ markings, mark=between positions 0.125 and 1 step 0.25 with {\arrow[line width=1]{stealth}}, }, postaction={decorate}]
			(1.7,-0.0) -- (2.0,+0.5) -- (2.6,-0.5) node [below left] {\footnotesize Guided ray} -- (2.9,+0.0);
		
		\draw [-stealth] (0.88,+0.5) -- (0.95, 0.65);
		\draw [-stealth] (1.22,+0.5) -- (1.29, 0.65);
		\draw [-stealth] (1.06,-0.5) -- (1.13,-0.65);
		\draw [-stealth] (1.38,-0.5) -- (1.45,-0.65);
		
	\end{axis}
\end{tikzpicture}
	\caption{Scheme of a dielectric slab waveguide, made by two materials with refractive indexes $n_H$ and $n_L$, with $n_H>n_L$.
		Two rays are shown: the unguided one is incident on the interface at an angle smaller than the critical angle.
		Conversely the guided ray is incident at a greater angle and is therefore totally reflected inside the waveguide \cite{Saleh1991}.}
	\label{fig:slabWG}
\end{figure}

2D-waveguides on the other hand are more common, to the point that with the term waveguide one usually refers to a device belonging to this category.
Since they confine light in two dimension, their core then becomes a straight path with which one can conduct light from a place to another.
A few types are shown in \autoref{fig:2DVG} below.

\begin{figure}[ht]
	\centering
%	\includegraphics[draft,width=9cm,height=6cm]{figures/foo.png}
	\includegraphics[scale=.4]{figures/WGtypes.pdf}
	\caption{Representation of a few types of 2D dielectric waveguides.
		Light is constrained in two direction and can only move forward or backward in the third direction.
		(a) strip, (b) embedded strip, (c) rib, (d) strip loaded \cite{Saleh1991}.
		}
	\label{fig:2DVG}
\end{figure}

Obviously, the versatility of straight (2D-) waveguides is limited.
Hence more complex structures such as bent waveguides have been developed.

\paragraph{Propagation of light inside waveguides\\}
\noindent Light propagates inside waveguides as superposition of transverse electro-magnetic (TEM) waves which keep reflecting on the interfaces of the core.
The result of the superposition is the so called \textit{mode}, which is expressed mathematically by
\begin{equation}
	\phi(\textbf{r}) = \phi_m\left( x, y \right) e^{i\left( \beta_m z - \omega t \right)}.
	\label{eq:mode_propagation}
\end{equation}
$\phi_m\left( x, y \right)$ is the transverse field distribution, $z$ is the direction of propagation, $\beta_m$ is the propagation constant of the m-th mode, and $\omega=2\pi \nu$ is the angular frequency of light.
Each mode, identified by the $m$ idex, travels inside the waveguide with a certain propagation constant $\beta$ and maintaining a field distribution $\phi_m$.
Both of them depend on the materials and the geometry of the waveguide, however while $\beta_m$ decrease with increasing $m$ indexes, the features of the function $\phi_m$ grows in number.
Light either propagate in a manner described by a mode or superposition of modes or is rapidly scattered away.

The filed distribution of some of the first modes, in the simplest case of a slab waveguide, is represented in \autoref{fig:WGmodes} below.
\begin{figure}[ht]
	\centering
	\includegraphics[scale=0.8]{figures/modes.png}\\
	\tikzsetexternalprefix{tikz/}	% set subfolder
\tikzsetnextfilename{WGmodes}
\begin{tikzpicture}[baseline]
	\begin{axis}[	width=\textwidth*0.75,
								height=200pt,
								axis lines=center,
								xlabel={$x$}, ylabel={$y$},
								xmin=-0.2, xmax=+4.0,
								ymin=-1.3, ymax=+1.3,
								domain=-1:1,
								xtick=\empty,
								ytick={-0.5,0.5},
								yticklabels={$-\dfrac{d}{2}$,$\dfrac{d}{2}$,},
								thick,
								area legend,
								legend style={draw=none},
								legend cell align=left,
								legend pos=outer north east,
								]
								
		\addlegendimage{black};
		\addlegendentry{$n_L$}
		\addplot [name path = upper, domain=0:3.8, 	forget plot] {+0.50};
		\addplot [name path = lower, domain=0:3.8,	forget plot] {-0.50};
		
		\addplot [pattern=crosshatch dots, pattern color=gray!50]
			fill between [ of=lower and upper];
		\addlegendentry{$n_H$}
		
		\addplot [black] ({0.5+.2    *exp(-8*x^2)}, {x});
		\addplot [black] ({1.0+.5*x  *exp(-8*x^2)}, {x});
		\addplot [black] ({1.5+2.*x^2*exp(-8*x^2)}, {x});
		\addplot [black] ({2.0+4.*x^3*exp(-8*x^2)}, {x});
%		
%		\draw [black] (0.5,-1.5) -- (3,-1.5);
%		\draw [black] (0.5,-1.5) -- (0.5,+1.5);
%		\draw [black] (3.0,-1.5) -- (3.0,+1.5);
%		\draw [black] (3.0,-1.5) -- (3.8,-0.9) -- (3.8,+2.1) -- (3,+1.5);
%		\draw [black] (0.5,+1.5) -- (1.3,+2.1) -- (3.8,+2.1);
%		
%		\path [name path= lowerFace, black] (3,-0.5) -- (3.8,+0.1);
%		\path [name path= upperFace, black] (3,+0.5) -- (3.8,+1.1);
%		\addplot [pattern=crosshatch dots, pattern color=gray!50]
%			fill between [ of=lowerFace and upperFace];
%		\draw [black] (3,-0.5) -- (3.8,+0.1) -- (3.8,+1.1) -- (3,+0.5);
%		
%		\draw [black, dashed] (1.3,+1.1) -- (3.8,+1.1);
%		\draw [black, dashed] (1.3,+1.1) -- (1.3,+2.1);
%		\draw [black, dashed] (1.3,+1.1) -- (0.5,+0.5);
%		
%		\draw [-stealth] (0,0) -- (0.4,0.3) node [above left] {$x$};
%		
%		\draw [	decoration={ markings, mark=between positions 0.0625 and 1 step 0.125 with {\arrow[line width=1]{stealth} }, }, postaction={decorate}]
%			(0.8,-0.0) -- (0.88,+0.5) -- (1.06,-0.5) node [below] {\footnotesize Unguided ray} -- (1.22,+0.5) -- (1.38,-0.5) -- (1.46,+0.0);
%			
%		\draw [	decoration={ markings, mark=between positions 0.125 and 1 step 0.25 with {\arrow[line width=1]{stealth}}, }, postaction={decorate}]
%			(1.7,-0.0) -- (2.0,+0.5) -- (2.6,-0.5) node [below left] {\footnotesize Guided ray} -- (2.9,+0.0);
				
	\end{axis}
\end{tikzpicture}
	\caption{Field distribution inside a slab waveguide.}
	\label{fig:WGmodes}
\end{figure}
The function inside the core is characterized by maxima, minima and nodes with a certain periodicity and the number of nodes (where the function is zero) is strictly linked to the index $m$.
Outside, the distribution of the field decays exponentially.
For waveguide with a 2D core, the field can be expected to be a multivariate version of the same or similar function.
However, differently from the analytical solution for slab waveguides, they are almost always found numerically.

The propagation constant can be written as
\begin{equation}
\beta_m = n_{eff} k_0 + i \alpha_{eff}
\end{equation}
where $k_0$ is the wavevector in vacuum, $n_{eff} \defeq c_0/v_{ph}$ is the \textit{effective refractive index}\index{effective refractive index} of the mode, and $\alpha_{eff}$ is \textit{effective absorption coefficient}\index{effective absorption coefficient}.
The effective refractive index is the ratio between the speed of light and the \textit{effective phase velocity} $v_{ph}$ at which each wavefront propagates in the core and its value is between the core refractive index and the cladding refractive index $n_L < n_{eff} < n_H$.

The physical interpretation of the effective index is that of an average refractive index \textit{felt} by the mode propagating as a whole, instead of a superposition of multiple TEM waves \textit{feeling} different refractive indexes depending on the specific medium in which they are propagating.
Similarly to the actual refractive index of materials, $n_{eff}$ is a complex number and the imaginary part is directly linked with the absorption properties of the materials.
Actually, a low imaginary part means that the material absorbs less and is therefore transparent to light.

All of these parameters which describe the propagation of light along the waveguide depends both on the materials and the geometry of the core and the cladding, but also on the frequency of light.
For dielectric waveguides, the fundamental mode ($m=0$) is always supported while higher modes might not be, depending both on the waveguides and on the frequency of light.
For this reason one distinguish \textit{single-mode} waveguides, which allow propagation for only the fundamental mode in a certain operative range of frequencies, from \textit{multi-mode} waveguides, which allow higher order modes to propagate.

Equation (\ref{eq:mode_propagation}) describes the propagation of a monochromatic light wave, which has ideally the same mode amplitude from $t=-\infty$ to $t=+\infty$.
This is obviously not a good physical representation, as light is generated and absorbed.
Eventually light is described to travel in wavepackets of finite duration, which are inherently non-monochromatic.

When light becomes non-monochromatic, the frequency dependence of the propagation constant $\beta$ has to be considered.
Usually, if working on a restricted range of frequencies, this dependence is characterized by a Taylor expansion at the first order.
\begin{subequations}
\begin{align}
	\beta\left(\omega\right)
		&\simeq \beta\left(\omega'\right) + \frac{\partial \beta\left(\omega\right) }{\partial \omega} \Delta\omega \label{eq:beta_taylor}\\
		&\simeq \dfrac{1}{v_{ph}} \omega' + \frac{1}{v_g} \Delta\omega \label{eq:beta_taylor_v}
\end{align}
\end{subequations}
where $1/v_{ph} = n_{eff}/c_0$, $\Delta\omega = \omega-\omega'$, and $v_g \defeq \left( \frac{\partial k}{\partial \omega}\right)^{-1}$.
The \textit{group velocity} $v_g$ describes the speed at which the ensemble of frequencies, the wavepacket, propagates.
Then, in a similar manner to $n_{eff}$ which is the ratio between $c_0$ and the effective phase velocity, one can define the \textit{effective group index} $n_{eff}^g$ as ratio between the speed of light in vacuum $c_0$ and the effective group velocity $v_g$ of a packet of light inside a waveguide:
\begin{equation}
	n_{eff}^g \defeq \dfrac{c_0}{v_g} = n_{eff}\left( \omega \right) +\omega\dfrac{\partial n_{eff}}{\partial \omega}.
	\label{eq:neff_group}
\end{equation}
The latter equation allow to describe the Taylor expansion of $\beta$ in other terms:
\begin{equation}
	\beta\left(\omega\right) \simeq \frac{n_{eff}}{c_0}\omega' + \frac{n_{eff}^g}{c_0} \Delta\omega
	\label{eq:beta_taylor_n}
\end{equation}
Nevertheless the monochromatic wave propagation model is still valid, because it is a very good approximation of slowly varying and almost monochromatic waves, such as the one I will use in my experiments.

\LARGE GVD? \normalsize

\subsection{Microring Optical Cavity}
\label{ssec:Microring_Optical_Cavity}
In integrated photonics the microring is an optical cavity made by bending a waveguide on itself.
To insert light into the cavity, the microring is often side coupled with one or two straight waveguides.
Light inserted from the different ports into the cavity interfere with itself, after a round trip, and with each other.
When the waves are in phase, they generate constructive interference and the energy stored in the cavity increases exponentially.
On the other hand, when the signals are out of phase, the interference is destructive and the energy stored in the cavity remains low.

%\begin{figure}[ht]
%	\centering
%%	\includegraphics[draft,width=9cm,height=6cm]{figures/foo.png}
%	\includegraphics[scale=.8]{figures/microring_theory.png}
%	\caption{•}
%	\label{fig:resonator_theory}
%\end{figure}

\begin{figure}[ht]
	\centering
	\begin{subfigure}[b]{0.3\textwidth}
		\centering
		\tikzsetexternalprefix{tikz/}	% set subfolder
\tikzsetnextfilename{APF_scheme}
\begin{tikzpicture}[baseline, thick, every node/.style={font=\scriptsize}]
	\draw [{stealth[reversed]}-stealth] (-1.5,0)
				node [below] {input} -- (1.5,0)
				node [below] {through};
	\draw [{stealth[reversed]}-stealth] (-0.5,0) -- (0.5,0);
	
	\draw (0,1) circle (0.9);
	\draw [-{stealth}] (0,1)
				-- ++(30:0.9cm) node [midway, below right] {R};
	
	\draw [-{stealth}] (0,0.1)
				arc [start angle=-90,end angle=-60, radius=0.9];
	\draw [-{stealth[reversed]}] (0,0.1)
				arc [start angle=-90,end angle=-120, radius=0.9];
	
	\draw [thin, densely dashed] (0,-0.2) -- (0,0.3);	
	
	\draw [thin]		(-0.5,0.4) -- (+0.5,0.4) --
								(+0.5,-0.3) -- (-0.5,-0.3) -- cycle;
\end{tikzpicture}
		\caption{All-Pass-Filter}
		\label{fig:APF_scheme}
	\end{subfigure}
	\hspace{.02\textwidth}
	\begin{subfigure}[b]{0.3\textwidth}
		\centering
		\tikzsetexternalprefix{tikz/}	% set subfolder
\tikzsetnextfilename{ADF_scheme}
\begin{tikzpicture}[baseline, thick, every node/.style={font=\scriptsize}]
	% lowe line & arrows
	\draw [{stealth[reversed]}-stealth] (-0.5,0) -- (+0.5,0);
	% lower long line
	\draw [{stealth[reversed]}-stealth] (-1.5,0)
				node [below] {input} -- (1.5,0)
				node [below] {through};
	% lower ring
	\draw [-{stealth}] (0,0.1)
				arc [start angle=-90,end angle=-60, radius=0.9];
	\draw [-{stealth[reversed]}] (0,0.1)
				arc [start angle=-90,end angle=-120, radius=0.9];
	% lower box
	\draw [thin, densely dashed] (0,-0.2) -- (0,0.3);	
	\draw [thin]		(-0.5,0.4) -- (+0.5,0.4) --
								(+0.5,-0.3) -- (-0.5,-0.3) -- cycle;
	\node at (0, 0.3) [above] {in};

	% circle
	\draw (0,1) circle (0.9);
	% radius arrow
	\draw [-{stealth}] (0,1)
				-- ++(30:0.9cm) node [midway, below right] {R};

	% upper lines & arrows
	\draw [{stealth[reversed]}-stealth] (+0.5,2) -- (-0.5,2);
	% upper long line
	\draw [{stealth[reversed]}-stealth] (+1.5,2)
				node [above] {add} -- (-1.5,2)
				node [above] {drop};
	% upper ring
	\draw [-{stealth[reversed]}] (0,1.9)
				arc [start angle=+90,end angle=+60, radius=0.9];
	\draw [-{stealth}] (0,1.9)
				arc [start angle=+90,end angle=+120, radius=0.9];
	% upper box
	\draw [thin, densely dashed] (0,2.2) -- (0,1.7);	
	\draw [thin]		(-0.5,1.6) -- (+0.5,1.6) --
								(+0.5,2.3) -- (-0.5,2.3) -- cycle;
	\node at (0, 2.2) [above] {out};
\end{tikzpicture}
		\caption{Add-Drop-Filter}
		\label{fig:ADF_scheme}
	\end{subfigure}
	\hspace{.02\textwidth}
	\begin{subfigure}[b]{0.3\textwidth}
		\centering
		\tikzsetexternalprefix{tikz/}	% set subfolder
\tikzsetnextfilename{APF-ADF_coupling_scheme}
\begin{tikzpicture}[baseline, line width=8pt, every node/.style={font=\small}]
	\draw [gray!30] (0,0) node [above, black] {$E^{r-}$}
				arc [start angle=-105,end angle=-75, radius=5] node [above, black] {$E^{r+}$};
	\draw [gray!30] (0,-0.8) node [below, black] {$E^{w-}$}
				-- (2.588190451,-0.8) node [below, black] {$E^{w+}$};

	\draw [thin, densely dashed] (1.294095225,-1.5) -- (1.294095225,.5);

	\draw [thin, dashed, {stealth[reversed]}-stealth, out=-15, in=180] (0,0)
				to node [pos=0.4, left]	{$i\kappa$} (2.588190451,-0.8);
	\draw [thin, dashed, {stealth[reversed]}-stealth, out=0, in=195] (0,-0.8)
				to node [pos=0.6, right]	{$i\kappa$} (2.588190451,-0.0);

	\draw [thin, dashed, {stealth[reversed]}-stealth] (0,0)
				arc [start angle=-105,end angle=-75, radius=5]
				node [midway, above right] {$\tau$};
	\draw [thin, dashed, {stealth[reversed]}-stealth] (0,-0.8)
				-- (2.588190451,-0.8) node [midway, below right] {$\tau$};
\end{tikzpicture}
		\caption{Coupling region}
		\label{fig:coupling_scheme}
	\end{subfigure}
	
	\caption{
		Schematic representation of microring configurations and of their coupling regions.
		APF has two channels, \textit{input} and \textit{through}, and one coupling region.
		ADF has four channels, \textit{input}, \textit{through}, \textit{add} and \textit{drop}, and two coupling regions.
	}
	\label{fig:resonator_theory}
\end{figure}

In the simpler case when there is only one waveguide side coupled to the resonator the optical cavity has only two ports, which are usually called \textit{input} and \textit{through}.
This configuration, shown in \autoref{fig:APF_scheme}, is called \textit{All-Pass Filter}, because in the ideal case, where no losses happen, all the signal passes from the input to the through channels.
%All pass filter theory\\
All pass resonators have only one input and one output: light then propagate in only in one direction, unless some specific phenomena happens (e.g. back scattering).

By adding a secondary waveguide coupled to the ring, one obtains the so called Add-Drop Filter (ADF) configuration, shown in \autoref{fig:ADF_scheme}.
Its name derive by the fact that the two additional ports are called \textit{add} and \textit{drop} respectively.
This simple structure can be readily used as a signal mixer: signals at the resonance wavelengths of the cavity are directed from the input to the drop channel or from the add channel to the through port.
Additionally signals out of resonance travel straight from the input to the through channel and from the add to the drop port.
Resonators with more than four ports are possible, however are very uncommon.

The theoretical model usually employed to describe resonators analytically decomposes their structure in a series of simpler substructures.
Both the APF and the ADF configurations are studied as ordered combinations of straight waveguides and bent waveguides, which are coupled together in specific \textit{coupling regions}, as shown in \autoref{fig:coupling_scheme}.
Other used descriptions, such as numerical simulation obtained with finite element methods (FEM), achieve more precise results, however they are more dependent on the specific geometry of the problem.
Moreover the approximation given by the analytical model is sufficiently accurate to make clear the physics behind it and it can be expanded to include non-trivial phenomena (see \autoref{ssec:Nonlinear_Perturbations}).
In the following section I will go through the necessary steps to solve the case of the ADF configuration in the theoretical model.

%1.2.3.2 Add-drop filter theory\\
\subsubsection{Add-Drop-Filter theory}
\label{sssec:Add-Drop-Filter_theory}
ADF configuration is obtained when a microring is coupled to two waveguides.
Such structure is composed by three different basic structures: four straight waveguides, two bent waveguides which together form the whole ring, and two coupling regions between the waveguides and the microring.
Each of these pieces transfers light to or from the outside or another piece.

Since our experiments are carried out in a time scale such that physics phenomena can be considered quasi-static, the field propagating in the device is described by a scalar complex function which, assuming a monochromatic continuous EM wave, loses any temporal dependence:
\begin{equation}
	E(z) = |E(z_0)|e^{i\beta z},
\end{equation}
where $z$ is the direction of propagation, loosely defined to accommodate propagation both along straight and bent waveguides ($z\sim r\theta$), $t$ is time, $\omega=2\pi\nu$ is the angular frequency, and $\beta = n_{eff}k_0+i\alpha_{eff}$ is the propagation constant.
Both dependence of the effective index $n_{eff}=n_{eff}\left(\omega\right)$ and the effective loss factor $\alpha_{eff}=\alpha_{eff}\left(\omega\right)$ will be expanded only when necessary.

Propagation in the straight waveguides is often considered without loss (i.e. $\alpha_{eff}\approx 0$) and thus neglected.
On the other hand, propagation in the two halves of the ring is frequently considered with radiative losses (i.e $\alpha_{eff}> 0$), because a bent waveguide is intrinsically more difficult to fabricate in comparison to a straight one.
The last structure type, the coupler between the resonator and the waveguides, is considered alike a beamsplitter.
In this approximation, light does not propagate in this part as it was in the previous ones.
Its operation is reduced to the exchange of power between the two input ports and two output ports, which is described by the following matrices:
\begin{equation}
\begin{pmatrix}
E^{w+}_{ch} \\
E^{r+}_{ch} \\
\end{pmatrix} = \textbf{M}
\begin{pmatrix}
E^{w-}_{ch} \\
E^{r-}_{ch} \\
\end{pmatrix}, \qquad \textbf{M} = 
\begin{pmatrix}
\tau & i\kappa \\
i\kappa & \tau \\
\end{pmatrix},
\end{equation}
where $E$ is the field amplitude of the `$_{ch}$' channel either in the waveguide `$^w$' or in the ring `$^r$', before `$^-$' and after `$^+$' the coupling.
An explanatory diagram is shown in \autoref{fig:coupling_scheme}.
The matrix $\textbf{M}$ is characterized by two real valued parameters, $\tau$ and $\kappa$, between $0$ and $1$.
Specifically, they must verify the following constraint:
\begin{equation}
\det\left(\textbf{M}\right) = |\tau|^2 + |i\kappa|^2 = 1.
\end{equation}
which represents the conservation of energy.

At this point one can solve the problem by putting all the pieces together.
Neglecting the propagation inside the waveguide, the full system of equation that describe the problem is:
\begin{subequations}
\begin{align}
E^{w+}_{in}	&= \tau 		 E^{w-}_{in} + i\kappa	E^{r-}_{in} \label{seq:in_w+}\\
E^{r+}_{in}	&= i\kappa	 E^{w-}_{in} + \tau 		E^{r-}_{in} \label{seq:in_r+}\\
\nonumber\\
E^{r-}_{out}	&= E^{r+}_{in}		e^{i\beta \pi R } \label{seq:out_r-_in}\\
E^{r-}_{in}	&= E^{r+}_{out}	e^{i\beta \pi R } \label{seq:in_r-_out}\\
\nonumber\\
E^{w+}_{out}	&= \tau 		 E^{w-}_{out} + i\kappa	E^{r-}_{out} \label{seq:out_w+}\\
E^{r+}_{out}	&= i\kappa	 E^{w-}_{out} + \tau 		E^{r-}_{out} \label{seq:out_r+}
\end{align}
\label{eq:ADF_equations}
\end{subequations}
The first two \cref{seq:in_w+,seq:in_r+} describe the exchange of optical power between the first channel and the microring resonator.
Then the third and fourth \cref{seq:out_r-_in,seq:in_r-_out} delineate the propagation of light in the two halves of the resonator, from one coupling region to the other.
Lastly the remaining two equation characterize the transfer of light between the resonator and the add-drop channel.

Using these equations it is easy to define new quantities of interest: the first one is transmittance from the \textit{input} to the \textit{through} port.
\begin{equation}
\eta_T\left(\omega\right) \defeq \dfrac{E^{w+}_{in}}{E^{w-}_{in}}
	= t\dfrac{1-e^{i\beta 2\pi R}}{1-\tau^2e^{i\beta 2\pi R}}
\end{equation}
Similarly, one can also define the transmittance from the \textit{input} to the \textit{drop} port.
\begin{equation}
\eta_D\left(\omega\right) \defeq \dfrac{E^{w+}_{out}}{E^{w-}_{in}}
	= \dfrac{-\kappa^2e^{i\beta\pi R}}{1-\tau^2e^{i\beta 2\pi R}}
\end{equation}

However, this quantities have complex values and therefore are difficult to study.
For this reason, one define also the transmission between the same ports as the square modulus of the transmittance.
Thus it follows that
\begin{equation}
T \left( \omega \right) \defeq |\eta_T|^2 = \tau 
\dfrac	{ \left( 1- 				\gamma \right)^2 + 4 				\gamma \sin^2 \left( n_{eff} k_0 \pi R \right) }
			{ \left( 1-\tau^2	\gamma \right)^2 + 4 \tau^2	\gamma \sin^2 \left( n_{eff} k_0 \pi R \right) }
\end{equation}
and
\begin{equation}
D \left( \omega \right) \defeq |\eta_D|^2 =
\dfrac	{ \kappa^4 \gamma}
			{ \left( 1-\tau^2	\gamma \right)^2 + 4 \tau^2	\gamma \sin^2 \left( n_{eff} k_0 \pi R \right) }
\end{equation}
where the notation is simplified by loss parameter $\gamma = e^{\pi R\alpha_{eff}}$.
The dependence of $T=T(\omega)$ and $D=D(\omega)$ from the frequency (or wavelength) of light is obtained by making explicit the wavevector dependence $k_0=\frac{\omega}{c_0}=\frac{2\pi}{\lambda}$ in which $c_0$ is the universal constant speed of light in vacuum.

\begin{figure}[!htbp]
	\centering
	\tikzsetexternalprefix{tikz/}	% set subfolder
\tikzsetnextfilename{ADF}
\begin{tikzpicture}[baseline]
	\newcommand\CENTRAL{193}
	\newcommand\START{193-5}
	\newcommand\STOP {193+7}
	\newcommand\CC{299792458}
	
	\begin{axis}[%
			axis x line*= bottom,
%			scaled x ticks = manual:{$+\SI{193}{\THz}$}{ \pgfmathparse{#1-\CENTRAL} },%
			xlabel = {Optical Frequency $\nu$ [$\si{\THz}$]},
			ylabel = {Transmission},
			legend columns=3,
			legend cell align=right,
			legend style={ at={(0.5,-0.22)}, anchor=north },
			/pgf/number format/1000 sep=,
			width=\textwidth*0.75,%
			height=207pt,
			xmin= 188, xmax = 199,
			%
			% global plot definition
			domain = \START:\STOP,
			samples = 551,
			smooth,
			no markers,
			cycle multi list={
%					exotic\nextlist 
					color list\nextlist
					solid,densely dotted
					%[2 of]linestyles
				},
			]
				
		\foreach \TA/\LA/\Radius/\Neff in {0.85/0.8/4/3.826, 0.85/0.8/5/3.826 }{ %, 0.9/0.99/5/3.826
			\edef\temp{
				\noexpand\addlegendimage{empty legend}
				\noexpand\addlegendentry{$\tau=\TA,\gamma=\LA,n_{eff}=\Neff, R=\SI{\Radius}{\um}:$};
				\noexpand\addplot 
										{ + (1-\TA^2)^2 * \LA
											/ ( (1-\TA^2 * \LA)^2 	+ 4 * \TA^2 	* \LA *sin(deg(\Neff*pi*\Radius*2*pi*x*1e6/\CC))^2 )
										};
				\noexpand\addlegendentry{$D(\omega)$};
				\noexpand\addplot 
										{ + \TA^2 * ( (1- \LA)^2	+ 4 					* \LA *sin(deg(\Neff*pi*\Radius*2*pi*x*1e6/\CC))^2 )
											/ ( (1-\TA^2 * \LA)^2 	+ 4 * \TA^2 	* \LA *sin(deg(\Neff*pi*\Radius*2*pi*x*1e6/\CC))^2 )
										};
				\noexpand\addlegendentry{$T(\omega)$};
								}
			\temp
		}
		
		\node [pin=45:{\scriptsize 76}] at (189.5568364, .346) {};
		\node [pin=45:{\scriptsize 77}] at (192.0510053, .346) {};
		\node [pin=45:{\scriptsize 78}] at (194.5451742, .346) {};
		\node [pin=45:{\scriptsize 79}] at (197.0393431, .346) {};
		
		\node [pin=45:{\scriptsize 61}] at (190.18038076,.346) {};
		\node [pin=45:{\scriptsize 62}] at (193.29809192,.346) {};
		\node [pin=45:{\scriptsize 63}] at (196.41580308,.346) {};
		
	\end{axis}
	
	\begin{axis}[
			width=\textwidth*0.75,%
			height=207pt,
			axis x line*= top,
			axis y line = none,
			xlabel = {Wavelength $\lambda$ [\si{\nm}]},
			xmin= 188, xmax = 199,
%			xmin= 150, xmax = 250,
%  		xtick = {187.37029,188.54872,189.74206,190.95061,192.17465,193.41449,194.67043,195.94278,197.23188,198.53805,199.86163},
			xtick = {187.37029,188.54872,189.74206,190.95061,192.17465,193.4143 ,194.67043,195.94278,197.2318 ,198.53805,199.86163},
%			xtick = {157.785504211, 171.309976, 187.37029, 199.86163, 214.13747, 230.609583077, 249.827048333},
%			scaled x ticks = manual:{$+\SI{1500}{\nm}$}{ \pgfkeys{/pgf/fpu}\pgfmathparse{(1e-3*\CC/#1) - 1500} },
			scaled x ticks = manual:{}{ \pgfkeys{/pgf/fpu}\pgfmathparse{(1e-3*\CC/#1)} },
			/pgf/number format/1000 sep=,
		]
		\addplot[black, opacity=0, domain = \START:\STOP] {0}; %
	\end{axis}
\end{tikzpicture}
	\caption{
		Transmission spectra of microring resonators in Add-Drop Filter configuration for the through and drop ports.
		The resonators considered differ in radius ($R=$\SI{4}{\um} and $R=$\SI{5}{\um}), but share the coupling constant $\tau=0.85$, the half round trip loss factor $\gamma=0.8$ and the effective index $n_{eff}=3.826$.
	}
	\label{fig:ADF}
\end{figure}

The transmission is much more interesting to study than the transmittance because it represents the ratio between the input and output optical powers ($I\propto |E|^2$), except for a constant factor, and is therefore a real valued function of $\omega$ instead of a complex one.
Both transmission spectra, as shown in \autoref{fig:ADF}, have either peaks or dips: the resonances of the microring.
Specifically, at each resonance, the through spectrum shows dips while the drop spectrum shows peaks.
The depth of the dips, the height of the peaks, and the width of both of them is defined by few parameters: the coupling constant $\tau$\footnote{one can choose also $\kappa$, but it does not matter which one, since $\tau^2 = 1-\kappa^2$} and the half round trip loss factor $\gamma = e^{-\pi R\alpha_{eff}}$.

The frequency of each resonance is identified by a positive integer number, that verifies the following equation:
\begin{equation}
	\omega_m = m\dfrac{c_0}{n_{eff}(\omega)R},
\end{equation}
where the dependence of $n_{eff}=n_{eff}\left(\omega\right)$ has been explicitly shown.
Moreover, it is easy to convert this quantity in other domains with:
\begin{equation}
	\omega = 2\pi\nu \qquad \mathrm{and} \qquad c_0 = \nu\lambda
	\label{eq:omega_nu_lambda}
\end{equation}
Each resonance is spaced from the next one by a quantity called \textit{Free Spectral Range}\index{Free Spectral Range (FSR)} $FSR_{\omega_m}$.
By exploiting the Taylor expansion of $\beta$ seen in \cref{eq:beta_taylor_n} one obtains
\begin{equation}
	FSR_{\omega_m} \simeq \dfrac{c_0}{n_{eff}^g\left(\omega_m\right) R} .
\end{equation}
where the \textit{effettive group index} $n_{eff}^g\left(\omega\right)$ has been defined in \cref{eq:neff_group}.
Another important quantity is the width of the peaks or dips of each resonance.
Ordinarily one evaluate the so called \textit{full-width-half-maximum} (FWHM):
\begin{equation}
	FWHM_{\omega_m} = \dfrac{c_0}{n_{eff}^g\left(\omega_m\right)}\dfrac{1- \tau^2\gamma}{\pi R \tau \sqrt{\gamma}}
\end{equation}

\begin{figure}[!hbtp]
	\centering
	\tikzsetexternalprefix{tikz/}	% set subfolder
\tikzsetnextfilename{ADF2}
\begin{tikzpicture}[baseline]
	\newcommand\CENTRAL{193}
	\newcommand\START{193-5}
	\newcommand\STOP {193+7}
	\newcommand\CC{299792458}
	
	\begin{axis}[%
			axis x line*= bottom,
%			scaled x ticks = manual:{$+\SI{193}{\THz}$}{ \pgfmathparse{#1-\CENTRAL} },%
			xlabel = {Optical Frequency $\nu$ [$\si{\THz}$]},
			ylabel = {Transmission},
			legend columns=3,
			legend cell align=right,
			legend style={ at={(0.5,-0.22)}, anchor=north },
			/pgf/number format/1000 sep=,
			width=\textwidth*0.75,%
			height=207pt,
			xmin= 188, xmax = 199,
			%
			% global plot definition
			domain = \START:\STOP,
			samples = 551,
			smooth,
			no markers,
			cycle multi list={
%					exotic\nextlist 
					color list\nextlist
					solid,densely dotted
					%[2 of]linestyles
				},
			]
				
		\foreach \TA/\LA/\Radius/\Neff in {0.85/0.9/5/3.826, 0.85/0.75/5/3.826 }{ %, 0.9/0.99/5/3.826
			\edef\temp{
				\noexpand\addlegendimage{empty legend}
				\noexpand\addlegendentry{$\tau=\TA,\gamma=\LA,n_{eff}=\Neff, R=\SI{\Radius}{\um}:$};
				\noexpand\addplot 
										{ + (1-\TA^2)^2 * \LA
											/ ( (1-\TA^2 * \LA)^2 	+ 4 * \TA^2 	* \LA *sin(deg(\Neff*pi*\Radius*2*pi*x*1e6/\CC))^2 )
										};
				\noexpand\addlegendentry{$D(\omega)$};
				\noexpand\addplot 
										{ + \TA^2 * ( (1- \LA)^2	+ 4 					* \LA *sin(deg(\Neff*pi*\Radius*2*pi*x*1e6/\CC))^2 )
											/ ( (1-\TA^2 * \LA)^2 	+ 4 * \TA^2 	* \LA *sin(deg(\Neff*pi*\Radius*2*pi*x*1e6/\CC))^2 )
										};
				\noexpand\addlegendentry{$T(\omega)$};
								}
			\temp
		}
		
		\draw [|<->|] (192.0446013229, 0.6) -- (194.5386870543, 0.6) node [midway, fill=white] {\scriptsize $FSR$};%0.5665693
	
		\draw [<-] (192.3, 0.283) -- (192.8,0.283) node [right] {\scriptsize $FWHM$}; %0.5665693/2 = 0.283
		\draw [<-] (191.8, 0.283) -- (191.3,0.283) {};
		
	\end{axis}
	
	\begin{axis}[
			width=\textwidth*0.75,%
			height=207pt,
			axis x line*= top,
			axis y line = none,
			xlabel = {Wavelength $\lambda$ [\si{\nm}]},
			xmin= 188, xmax = 199,
%			xmin= 150, xmax = 250,
%  		xtick = {187.37029,188.54872,189.74206,190.95061,192.17465,193.41449,194.67043,195.94278,197.23188,198.53805,199.86163},
			xtick = {187.37029,188.54872,189.74206,190.95061,192.17465,193.4143 ,194.67043,195.94278,197.2318 ,198.53805,199.86163},
%			xtick = {157.785504211, 171.309976, 187.37029, 199.86163, 214.13747, 230.609583077, 249.827048333},
%			scaled x ticks = manual:{$+\SI{1500}{\nm}$}{ \pgfkeys{/pgf/fpu}\pgfmathparse{(1e-3*\CC/#1) - 1500} },
			scaled x ticks = manual:{}{ \pgfkeys{/pgf/fpu}\pgfmathparse{(1e-3*\CC/#1)} },
			/pgf/number format/1000 sep=,
		]
		\addplot[black, opacity=0, domain = \START:\STOP] {0}; %
	\end{axis}
\end{tikzpicture}
	\caption{Transmission spectra of microring resonators in Add-Drop Filter configuration for the through and drop ports.
	The microring has a radius of \SI{5}{\um} and an effective index $n_{eff}=3.826$.
	The coupling constant is $\tau=0.85$, while the half round trip loss factor $\gamma$ takes the two values of $0.9$ and $0.75$.
	The arrows indicate a $FSR_\nu \approx \SI{2.49}{\THz}$ and a $FWHM_\nu \approx \SI{0.3}{\THz}$, which gives a quality factor of around $Q \approx \num{650}$.
	% aggiungere esempi di Qfactor
	}
	\label{fig:ADF2}
\end{figure}

Values such this are often expressed in the wavelength domain: however to obtain the $FSR_\lambda$ one can not use \cref{eq:omega_nu_lambda}.
Instead the following relations have to be used:
\begin{equation}
	\Delta\omega = 2\pi \Delta \nu = \dfrac{2\pi c_0}{\lambda^2}\Delta\lambda
		\qquad \mathrm{and} \qquad
	\Delta\lambda = \dfrac{c_0}{\nu^2}\Delta\nu = \dfrac{2\pi c_0}{\omega^2}\Delta\omega
\end{equation}
By employing the first relation, one obtains
\begin{equation}
	FSR_{\lambda_m} \simeq \dfrac{\lambda_m^2}{n_{eff}^g\left(\lambda_m\right) 2\pi R}
		\qquad \mathrm{and} \qquad
	FWHM_{\lambda_m} \simeq	\dfrac{\lambda_m^2}{n_{eff}^g}
													\dfrac{1- \tau^2\gamma}{2\pi^2 R \tau \sqrt{\gamma}}
\end{equation}
where $n_{eff}^g\left(\lambda_m\right)$ is an appropriate redefinition of the effective group index in the wavelength domain.

An additional figure of merit of an optical microring resonator is the \textit{quality factor} $Q-factor$, or simply $Q$.
The quality factor has many similar definitions, depending on the field of study.
In the most physical, but less operative one defines it as $2\pi$ times the ratio between the energy stored in the cavity and the energy lost each cycle.
$$Q \defeq 2\pi \times \frac{\mathrm{energy~stored}}{\mathrm{energy~lost~per~cycle}}$$
However, for my purposes this definition is too cumbersome, thus a similar one is used instead:
\begin{equation}
Q \defeq \dfrac{\omega_m}{FWHM_{\omega_m}} = \dfrac{\lambda_m}{FWHM_{\lambda_m}}
\end{equation}
Both are sensible definitions and as $Q$ becomes larger they become approximately equivalent.

\begin{figure}
	\centering
	\begin{subfigure}[t]{0.47\textwidth}
		\centering
%		\includegraphics[draft,width=8cm,height=6cm]{figures/foo.png}
		\tikzsetexternalprefix{tikz/}	% set subfolder
\tikzsetnextfilename{ADF_D_contour}
\begin{tikzpicture}[baseline]
	
	\begin{axis}[
			title = {$D\left( \omega_{m} \right)$},
			width=8cm, % height=207pt,
			domain = 0:1,
			xmax=1,
			xlabel = {$\tau^2$},
			ylabel = {$\gamma$},
			samples = 101,
			view={0}{90},
			]
		
		\addplot3 [contour gnuplot={
								labels over line,
								levels={0.01,0.05,0.1,0.25,0.5,0.75,0.9},
								handler/.style=smooth,
								label distance=150pt,
								contour label style={
										/pgf/number format/fixed,
										/pgf/number format/precision=2,
									},
								}
		] {(1-x)^2*y/(1-x*y)^2	};
	\end{axis}

%%% loglog contour plot	
%	\begin{axis}[
%			title = {$D\left( \nu_{m} \right)$},
%			width=8cm, % height=207pt,
%			domain = -3:0,
%			ticks=none,
%			yticklabel pos=upper,
%			samples = 51,
%			smooth,
%			view={0}{90},
%			]
%		
%		\addplot3 [contour gnuplot={
%								labels over line,
%%								levels={ 2e4,4e4,6e4,1e5,2e5,6e5,1e6},
%								handler/.style=smooth,
%								}
%		] {(1-10^x)^2*10^y/(1-10^(x+y))^2	};
%		
%	\end{axis}
%
%	\begin{loglogaxis}[
%			width=8cm, % height=207pt,
%			xlabel = {$\tau^2$},
%			ylabel = {$\gamma$},
%			yticklabel pos=right,
%			domain = 1e-3:1e0,
%			xmax = 1e0, xmin=1e-3,
%			ymax = 1e0, ymin=1e-3,
%		]
%		\addplot [opacity=0] {x};
%	\end{loglogaxis}
\end{tikzpicture}
		\caption{Maximum transmission $D\left( \omega_m \right)$}
		\label{fig:D_contour}
	\end{subfigure}
	\hspace{0.03\textwidth}
	\begin{subfigure}[t]{0.47\textwidth}
		\centering
%		\includegraphics[draft,width=8cm,height=6cm]{figures/foo.png}
		\tikzsetexternalprefix{tikz/}	% set subfolder
\tikzsetnextfilename{ADF_Q_contour}
\begin{tikzpicture}[baseline]

	\begin{axis}[
			title = {$Q$},
			width=8cm, % height=207pt,
			domain = 0:1,
			xmin = 0, xmax = 1,
			ymin = 0, ymax = 1,
			xlabel = {$\tau^2$},
			ylabel = {$\gamma$},
			yticklabel pos=upper,
			samples = 51,
			smooth,
			view={0}{90},
%			view={30}{30},
			]
		
%		\addplot3 [surf] { 2*pi*200e12*(4*pi*5e-6*sqrt(x*y))/(299792458*(1-x*y)) };	
		\addplot3 [contour gnuplot={
								labels over line,
								levels={5e1,1e2,2e2,4e2,1e3},
								handler/.style=smooth,
								label distance=50pt,
								}
		]  { 2*pi*200e12*(4*pi*5e-6*sqrt(x*y))/(299792458*(1-x*y)) };	
		
	\end{axis}

%%% loglog contour plot
%	\begin{axis}[
%			title = {$Q$},
%			width=8cm, % height=207pt,
%			domain = -3:0,
%			ticks=none,
%			yticklabel pos=upper,
%%			samples = 51,
%			smooth,
%			view={0}{90},
%			]
%		
%		\addplot3 [contour gnuplot={
%								labels over line,
%%								levels={ 2e4,4e4,6e4,1e5,2e5,6e5,1e6},
%								levels={ 20000., 32613.78817907, 53182.95896945, 86724.88792828,
%													141421.35623731, 230614.30781599, 376060.30930864,
%													613237.5635173, 1000000
%									},
%								handler/.style=smooth,
%								}
%%		] { (1-x*y)/(4*pi*5e-6*sqrt(x*y)) };
%		] { (1-10^(x+y))/(10*pi*5e-6*sqrt(10^(x+y))) };		
%		
%	\end{axis}
%
%	\begin{loglogaxis}[
%			width=8cm, % height=207pt,
%			xlabel = {$\tau^2$},
%			ylabel = {$\gamma$},
%			yticklabel pos=right,
%			yticklabel=\ ,
%			domain = 1e-3:1e0,
%			xmax = 1e0, xmin=1e-3,
%			ymax = 1e0, ymin=1e-3,
%		]
%		\addplot [opacity=0] {x};
%	\end{loglogaxis}
\end{tikzpicture}
		\caption{Quality factor $Q$}
		\label{fig:Q_contour}
	\end{subfigure}
	\caption{Contour plots of the maximum transmission on the drop channel $D\left( \omega_m \right)$ and of the quality factor $Q$, parametrized for the coupling coefficient $\tau^2$ and the half round trip loss factor $\gamma$.
	The $Q-factor$ is obtained for a resonance near $v_m\approx\SI{200}{\THz}$ (arbitrarily chosen) and an effective group index $n_{eff}^g \approx 4$.
	}
	\label{fig:APF_contour_plots}
\end{figure}

\subsubsection{Critical coupling}
\label{sssec:Critical_coupling}
Critical coupling/ over-coupling / under-coupling / enhancement factor

%\subsection{Light coupling}
%\label{ssec:light_coupling}
We covered how light travels inside a waveguide, now the problem on how to insert and extract light from it.
The process of inserting light into a waveguide is called \textit{coupling} and it often requires at the same time to extract light from another waveguide.

%subsubsection{End coupling}
%label{sssec:end_coupling}
The most naive way to insert light into a waveguide is obviously to use one of its ends.
This is achieved either by radiating a beam directly on the whole waveguide or by focusing it inside the core with some lensing mechanism (e.g. tapered optical fibers).

These methods are very simple, however they have a low coupling efficiency. % and are only able to access structures at perimeter of a chip.

%\subsubsection{Grating coupling}
%\label{sssec:grating_coupling}
% and prism coupling?
A more sophisticated way to couple light inside a waveguide is to use the so called \textit{grating coupler}.
A grating coupler is a periodic structure end coupled to a waveguide.
It is designed in a way such that light incident from free space above it, at a certain angle and with a specific wavelength, is coupled inside the waveguide.

%\subsubsection{Evanescent coupling}
%\label{sssec:evanescent_coupling}
Evanescent coupling exploits the fact that the evanescent field distribution outside the core decays exponentially to zero \cite{Reed2004}.
This means that when two waveguides are separated by a sufficiently small distance, the field distribution of one waveguide cannot decays sufficiently fast to zero and instead extends itself over the core of the second waveguide.
When this happens optical power can be transferred between the waveguides.
Therefore, if at the beginning light is propagating inside one waveguide, light is \textit{coupled} inside the other waveguide because its mode overlaps the core of the second waveguide.
Depending on the parameters of the problems, light can couple completely or partially between the waveguides.

This exchange of optical power is periodic with their length.

\subsection{Nonlinear Perturbations}
\label{ssec:Nonlinear_Perturbations}
In the precedent section, I considered the material of waveguides and resonators as a linear medium.
This is however only an approximation, because almost any material shows some sort of nonlinearity if probed with a high enough electromagnetic field.

In integrated photonics, due to the lateral confinement of the field inside the waveguide, much higher field amplitude are reached in comparison to free space.
Moreover, inside an optical cavity, such as an appropriately built microring resonator, the field reaches even more higher amplitudes thanks to its large enhancement factor and nonlinearities might arise.

Silicon response to light is almost linear.
However, a silicon optical cavity, by confinement and enhancement, can obtain a very high electromagnetic field inside, such that its response becomes significantly nonlinear.

One can distinguish two kinds of optical nonlinearities shown by silicon: electronic nonlinearities and thermal nonlinearities.

% ### refractive index
% nSi = 3.48           # Silicon refractive index
% n0 = nSi             # standard refractive index
% n2 = 5e-14           # [1/(W/cm²)] intensity-dependent refractive index
% n2 = 4.5e-18         # [1/(W/m²)]  intensity-dependent refractive index
% dndT = 1.86e-4       # [1/K]
% dndN = -1.73e-27     # [m³]
% dαdN =  1.1e-15      # [m²]
% βtpa =  0.79e-11      # [m/W]
% vg = c0/4.0          # [m/s]

\subsubsection{Electronic nonlinearities}
\label{sssec:Electronic_nonlinearities}
Silicon is a centrosymmetric material and thus does not exhibit optical nonlinearities of even orders.
The most efficient nonlinear effects belong to the third order.

Silicon nonlinearities are mainly due to two fundamental effects of this category and their correlated effects.
\paragraph{Kerr effect}
The first one is the Kerr effect, which is given by the real part of the third order nonlinear susceptibility $\chi^{(3)}$ and it produces a change in the refractive index characterized by:
\begin{equation}
	\Delta n_{Kerr} = n_2 I,
\end{equation}
where $I$ is the intensity of light beam and $n_2$ is the second-order (or intensity dependent) nonlinear refractive index, defined by
\begin{equation}
	n_2 = \dfrac{3}{4\varepsilon_0 n_0^2 c_0} \mathcal{R}\mathrm{e} \left[ \chi^{(3)} \right]
	\overset{Si}{\simeq} \SI{0.45e-17}{\square\m\per\W}.
\end{equation}

\paragraph{Two photon absorption}
The other fundamental effect is the two photon absorption (TPA), which is linked instead to the imaginary part of the third order nonlinear susceptibility.
It does not produce a change to the real part of the refractive index but to its imaginary part, which is in turn linked to the absorption coefficient by $\alpha=2\mathcal{I}\mathrm{m}[n]\omega/c_0$:
\begin{equation}
	\Delta\alpha_{TPA} = \beta_{TPA} I,
\end{equation}
where $\beta_{TPA}$ is the TPA coefficient which is experimentally found to be
\begin{equation}
	\beta_{TPA} \overset{Si}{\simeq} \SI{0.79e-11}{\m\per\W}.
\end{equation}

As a consequence to the absorption of two photons, a free carrier is generated in the conduction band of silicon.
The presence of free carriers alters the refractive index in its real and imaginary parts as well with  two additional effects.

\subparagraph{Free carrier effects}
When free carriers are generated in the conduction band, they can either move around until they thermalize or they can absorb an incoming photon.
The first case is the \textit{free carrier dispersion} (FCD) effect, which affects the real part of the refractive index.
The second case is the \textit{free carrier absorption} (FCA) effect and affects the imaginary part of the refractive index.
Both effects are usually characterized as first order expansion, as follows:
\begin{align}
	n			\left(N	\right) &= n				\left(N_0\right) + \left.\dfrac{dn}{dN}\right|_{N_0} \Delta N \\
	\alpha	\left(N	\right) &= \alpha	\left(N_0\right) + \left.\dfrac{d\alpha}{dN}\right|_{N_0} \Delta N
\end{align}
where the value of the FCD and FCA coefficients is
\begin{equation}
	\left.\dfrac{dn}{dN}\right|_{N_0} \overset{Si}{\simeq} \SI{-1.73e-21}{\cubic\m}
	\qquad and \qquad
	\left.\dfrac{d\alpha}{dN}\right|_{N_0} \overset{Si}{\simeq} \SI{1.1e-15}{\square\m}
\end{equation}
respectively.

\subsubsection{Thermal nonlinearities}
\label{sssec:Thermal_nonlinearities}
The thermal nonlinearity of materials refractive index linked to their optical response is called \textit{thermo-optic effect} (TOE).
When light propagates inside a medium, a portion of the photons is absorbed.
Hence the material heats up and its refractive index changes.
To characterize this change usually a first order expansion is made:
\begin{equation}
	n\left(T\right) = n\left(T_0\right) + \left.\dfrac{dn}{dT}\right|_{T_0} \Delta T,
	\label{eq:TOC}
\end{equation}
where $\dfrac{dn}{dT}$ is called \textit{thermo-optic coefficient} and $\Delta T$ is the temperature shift.

In silicon waveguides, the shift in temperature is caused both by linear and nonlinear processes.
The linear process is characterized by the linear absorption coefficient $\alpha$.
Similarly, the nonlinear processes of heat generation are linked to the nonlinear part of the absorption coefficient and are given mainly by the two photon absorption and by the free carrier absorption.

The thermo-optic coefficient (TOC) of silicon at \SI{300}{\K} is \cite{??} :
\begin{equation}
	\left.\dfrac{dn}{dT}\right|_{\SI{300}{\K}} = \SI{1.86e-4}{\per\K}.
	\label{eq:Si_TOC}
\end{equation}
Moreover, due to the fact that in silicon integrated structures the cladding is usually made by $SiO_2$, a thermally insulating material, the heat generated by the beam confined in the cavity will stay within the core, thus amplifying the effect even further.

\begin{figure}[!htbp]
	\centering
	\includegraphics[draft,width=9cm,height=6cm]{figures/foo.png}
	\caption{•}
	\label{fig:NL_effects}
\end{figure}

\section{Integrated photonics applied to ANNs}
\label{sec:Integrated_photonics_applied_to_ANNs}
Since our experiments are carried out in a time scale such that physics phenomena can be considered quasi-static, we obtain also that the only nonlinear effect non negligible is the thermo-optic effect.

\subsection{Weighted sum of inputs}
\label{ssec:Weighted_Sum_of_inputs}
This has already been demonstrated and integrated widely, so it will not be the focus of this work.
Two example are the banks of microring resonators and the MZ interferometers.

\subsection{Nonlinear Activation Function}
\label{ssec:Nonlinear_Activation_Function}
As opposed to the mechanism for weighted sum, an optical phenomenon for the activation function in an integrated photonic circuit has yet to be proposed.

\subsection{Simulations}
\label{ssec:Simulations}

\begin{figure}[ht]
	\centering
	\tikzsetexternalprefix{tikz/}	% set subfolder
\tikzsetnextfilename{bistability_test}
% This file was created by matplotlib2tikz v0.6.15.
\begin{tikzpicture}
	
	\pgfplotstableread[col sep=tab, header=true]{tikz/foo.csv}\loadedtable
	
	\pgfplotstablegetcolsof{\loadedtable}
	\pgfmathparse{\pgfplotsretval - 1}
	\newcommand{\ncol}{\pgfmathresult}
	
	\definecolor{color1}{rgb}{1,0.498039215686275,0.0549019607843137}
	\definecolor{color6}{rgb}{0.890196078431372,0.466666666666667,0.76078431372549}
	\definecolor{color5}{rgb}{0.549019607843137,0.337254901960784,0.294117647058824}
	\definecolor{color2}{rgb}{0.172549019607843,0.627450980392157,0.172549019607843}
	\definecolor{color4}{rgb}{0.580392156862745,0.403921568627451,0.741176470588235}
	\definecolor{color3}{rgb}{0.83921568627451,0.152941176470588,0.156862745098039}
	\definecolor{color0}{rgb}{0.12156862745098,0.466666666666667,0.705882352941177}

\begin{axis}[
		title={Internal Power},
		xlabel={Pump Power $P$ $[mW]$},
		ylabel={Internal Power [a.u.*]},
		xmin=0.852408771935031, xmax=4.09941578936435,
		ymin=-3215.6821653405, ymax=90671.1511338174,
		tick align=outside,
		tick pos=left,
		width=\textwidth*0.75,%
		height=207pt,
		%x grid style={lightgray!92.02614379084967!black},
		%y grid style={lightgray!92.02614379084967!black},
		legend entries={
			{$\Delta\lambda$},
			{\SI{350}{\pm}},
			{\SI{379}{\pm}},
			{\SI{407}{\pm}},
			{\SI{436}{\pm}},
			{\SI{464}{\pm}},
			{\SI{493}{\pm}},
			{\SI{521}{\pm}},
			{\SI{550}{\pm}}
			},
		legend pos = outer north east,
		%legend cell align={left},
		%legend style={at={(1,-0.1)}, anchor=north west, draw=white!80.0!black},
		%legend columns=8
	]
	
	\addlegendimage{empty legend}
	\addlegendimage{mark=*, color0}
	\addlegendimage{mark=*, color1}
	\addlegendimage{mark=*, color2}
	\addlegendimage{mark=*, color3}
	\addlegendimage{mark=*, color4}
	\addlegendimage{mark=*, color5}
	\addlegendimage{mark=*, color6}
	\addlegendimage{mark=*, gray!99.6078431372549!black}

%\addlegendentry = {$\Delta\lambda$}
%\addlegendimage = {empty legend}
%\foreach \i in {1,2...,\pgfplotsretval-1} {
%	\addplot table [x index=0, y index=\i] {\loadedtable};
%	\pgfplotstablegetcolumnnamebyindex{\i}\of{\loadedtable}\to\colname
%	\addlegendentry = {\SI{\colname}{\pm}}
%	\addlegendimage = {mark=*, color\i}
%}

\addplot [semithick, color0, mark=*, mark size=1, mark options={solid}]
table {%
1 2182.13784798453
1.10996151344182 2732.06812562319
1.20997071148193 3301.35802903992
1.30232241935941 3891.75515464847
1.38854537026638 4505.26943967214
1.46971861477299 5144.22980210166
1.54663743906977 5811.35713826561
1.61990800024395 6509.85972553239
1.69000487886171 7243.55987029429
1.75730789900303 8017.0649998584
1.82212667320922 8836.00339261668
1.88471753176505 9707.35631717665
1.94529553946556 10639.9381155571
2.00404323735458 11645.110858216
2.06111713847338 12737.8851497554
2.11665264505793 13938.6847549387
2.1707678321628 15276.31091276
2.22356640163383 16793.2014266478
2.27514001850367 18555.3533357662
2.32557018064795 20672.1913566021
2.37492973084278 23336.6084050299
2.42328409142697 26875.9830526321
2.47069228134243 31476.9310455102
2.51720776067593 36046.2694806532
2.56287913716765 39544.1852080518
2.60775076129818 42181.9936544263
2.65186323070653 44282.0696014857
2.69525382027284 46032.9554507758
2.73795685083191 47541.7982041075
2.78000400689269 48873.3912118438
2.82142461172746 50069.4947595316
2.86224586662004 51158.5039921599
2.9024930598204 52160.5856929643
2.94218974976591 53090.5802810476
2.98135792633989 53959.7327474007
3.02001815330168 54776.7742586017
3.05818969450781 55548.625360946
3.09589062612293 56280.8689540435
3.13313793667592 56978.0778524274
3.16994761653247 57644.0474967912
3.20633473812108 58281.9649496754
3.24231352805431 58894.5341847922
3.27789743212414 59484.0704107321
3.31309917401337 60052.5723212442
3.34793080845022 60601.7782354623
3.38240376943561 61133.2101736674
3.41652891409018 61648.2089375426
3.45031656259774 62147.9622831466
3.48377653466154 62633.5277873525
3.51691818283855 63105.8515562808
3.54975042307237 63565.7836494865
3.58228176270719 64014.090916815
3.61452032623231 64451.4677214149
3.64647387897796 64878.5449862339
3.67814984895812 65295.8978439847
3.70955534703456 65704.052169585
3.74069718555709 66103.4901635271
3.77158189561844 66494.655157941
3.80221574304758 66877.9557805134
3.83260474325241 67253.7695572549
3.86275467501147 67622.4460626382
3.89267109330405 67984.3096727404
3.92235934125968 68339.6619855259
3.95182456129939 68688.7839488388
};
\addplot [semithick, color1, mark=*, mark size=1, mark options={solid}]
table {%
1 1940.07544411095
1.10996151344182 2423.22716305164
1.20997071148193 2920.70721201128
1.30232241935941 3433.61186262683
1.38854537026638 3963.17696183345
1.46971861477299 4510.8034528873
1.54663743906977 5078.08911859356
1.61990800024395 5666.86848769019
1.69000487886171 6279.26358787269
1.75730789900303 6917.74933289089
1.82212667320922 7585.2389829278
1.88471753176505 8285.19764576196
1.94529553946556 9021.7957820453
2.00404323735458 9800.12113436134
2.06111713847338 10626.4782986138
2.11665264505793 11508.8239192492
2.1707678321628 12457.4194811703
2.22356640163383 13485.848418119
2.27514001850367 14612.6752996465
2.32557018064795 15864.3096818447
2.37492973084278 17280.3143272715
2.42328409142697 18924.1988617378
2.47069228134243 20908.2895226999
2.51720776067593 23462.0392880969
2.56287913716765 27170.8381037511
2.60775076129818 33631.0297137638
2.65186323070653 40341.9479641702
2.69525382027284 44189.4013634534
2.73795685083191 46803.6856546889
2.78000400689269 48821.2416782276
2.82142461172746 50486.4792493101
2.86224586662004 51917.3882993097
2.9024930598204 53180.0763622241
2.94218974976591 54315.4654194578
2.98135792633989 55350.7465285369
3.02001815330168 56304.9665061537
3.05818969450781 57192.0231155222
3.09589062612293 58022.3918927019
3.13313793667592 58804.1805477364
3.16994761653247 59543.8034465492
3.20633473812108 60246.430206525
3.24231352805431 60916.2935960542
3.27789743212414 61556.9070656952
3.31309917401337 62171.2219851855
3.34793080845022 62761.7437447084
3.38240376943561 63330.6190515937
3.41652891409018 63879.7026494471
3.45031656259774 64410.6090421554
3.48377653466154 64924.7531439948
3.51691818283855 65423.3825577898
3.54975042307237 65907.6035338806
3.58228176270719 66378.4020054886
3.61452032623231 66836.6608209025
3.64647387897796 67283.1739473754
3.67814984895812 67718.6582859319
3.70955534703456 68143.7635642359
3.74069718555709 68559.0806667746
3.77158189561844 68965.1487014138
3.80221574304758 69362.4610174769
3.83260474325241 69751.4703637162
3.86275467501147 70132.5933346228
3.89267109330405 70506.2142143482
3.92235934125968 70872.6883067225
3.95182456129939 71232.3448615192
};
\addplot [semithick, color2, mark=*, mark size=1, mark options={solid}]
table {%
1 1732.97280209881
1.10996151344182 2160.25082035293
1.20997071148193 2598.25542831804
1.30232241935941 3047.68385058352
1.38854537026638 3509.3086372655
1.46971861477299 3983.98931489536
1.54663743906977 4472.68642687737
1.61990800024395 4976.47859191022
1.69000487886171 5496.5834020755
1.75730789900303 6034.38326074926
1.82212667320922 6591.45765894542
1.88471753176505 7169.62394084324
1.94529553946556 7770.98943472775
2.00404323735458 8398.01903253042
2.06111713847338 9053.62415924678
2.11665264505793 9741.28193834583
2.1707678321628 10465.1979664045
2.22356640163383 11230.5337188335
2.27514001850367 12043.7326151105
2.32557018064795 12913.002021987
2.37492973084278 13849.0520482085
2.42328409142697 14866.2785935804
2.47069228134243 15984.7629825415
2.51720776067593 17233.8916436386
2.56287913716765 18659.5263921909
2.60775076129818 20340.0938776341
2.65186323070653 22430.0792057072
2.69525382027284 25321.8960673326
2.73795685083191 30987.8016123129
2.78000400689269 44888.9927766667
2.82142461172746 48918.4586028915
2.86224586662004 51428.5184753255
2.9024930598204 53337.2501195886
2.94218974976591 54909.1358036994
2.98135792633989 56261.2528516411
3.02001815330168 57456.8190350748
3.05818969450781 58534.2614456403
3.09589062612293 59518.8696435552
3.13313793667592 60428.26786974
3.16994761653247 61275.282136951
3.20633473812108 62069.5689773745
3.24231352805431 62818.6003117727
3.27789743212414 63528.2893700834
3.31309917401337 64203.4049985116
3.34793080845022 64847.8553330121
3.38240376943561 65464.8876903212
3.41652891409018 66057.2328970549
3.45031656259774 66627.2118303121
3.48377653466154 67176.8154208414
3.51691818283855 67707.7658504522
3.54975042307237 68221.5639803765
3.58228176270719 68719.5266349635
3.61452032623231 69202.8162397473
3.64647387897796 69672.4646422314
3.67814984895812 70129.3924540031
3.70955534703456 70574.4248816071
3.74069718555709 71008.3048010219
3.77158189561844 71431.7036445875
3.80221574304758 71845.2304972518
3.83260474325241 72249.4397994264
3.86275467501147 72644.8378492904
3.89267109330405 73031.8883777777
3.92235934125968 73411.0173027476
3.95182456129939 73782.6168527726
};
\addplot [semithick, color3, mark=*, mark size=1, mark options={solid}]
table {%
1 1555.12712962575
1.10996151344182 1935.35022328765
1.20997071148193 2323.70064537386
1.30232241935941 2720.62917693881
1.38854537026638 3126.62808241723
1.46971861477299 3542.23655447464
1.54663743906977 3968.04710570885
1.61990800024395 4404.71310972971
1.69000487886171 4852.95775950785
1.75730789900303 5313.58477279263
1.82212667320922 5787.49128427383
1.88471753176505 6275.68348806087
1.94529553946556 6779.2957771205
2.00404323735458 7299.61438270963
2.06111713847338 7838.1068655017
2.11665264505793 8396.45931641251
2.1707678321628 8976.62387064854
2.22356640163383 9580.88022625359
2.27514001850367 10211.916530394
2.32557018064795 10872.9375886795
2.37492973084278 11567.812506292
2.42328409142697 12301.2807338782
2.47069228134243 13079.2472517646
2.51720776067593 13909.2186734183
2.56287913716765 14800.9715720034
2.60775076129818 15767.623259535
2.65186323070653 16827.4446215626
2.69525382027284 18007.1530903572
2.73795685083191 19348.4800019746
2.78000400689269 20923.1011850703
2.82142461172746 22874.1137116357
2.86224586662004 25581.679254139
2.9024930598204 32373.8833560084
2.94218974976591 53685.3764633791
2.98135792633989 56034.1288896655
3.02001815330168 57820.9838468359
3.05818969450781 59297.9114358345
3.09589062612293 60573.1864351054
3.13313793667592 61704.6888586915
3.16994761653247 62727.4816976698
3.20633473812108 63664.6266831185
3.24231352805431 64532.2068880532
3.27789743212414 65341.9425439077
3.31309917401337 66102.6722811745
3.34793080845022 66821.2474128613
3.38240376943561 67503.0999560958
3.41652891409018 68152.6186366019
3.45031656259774 68773.406464817
3.48377653466154 69368.4623655541
3.51691818283855 69940.3124659625
3.54975042307237 70491.1070517066
3.58228176270719 71022.6935267439
3.61452032623231 71536.6722322788
3.64647387897796 72034.4397820577
3.67814984895812 72517.2231554371
3.70955534703456 72986.1068373264
3.74069718555709 73442.0546571947
3.77158189561844 73885.9275378199
3.80221574304758 74318.4980334294
3.83260474325241 74740.4623689852
3.86275467501147 75152.4504450848
3.89267109330405 75555.034238381
3.92235934125968 75948.7348944082
3.95182456129939 76334.028743014
};
\addplot [semithick, color4, mark=*, mark size=1, mark options={solid}]
table {%
1 1401.7392420467
1.10996151344182 1742.05447923165
1.20997071148193 2088.60129970629
1.30232241935941 2441.67621144985
1.38854537026638 2801.59908560004
1.46971861477299 3168.71577477143
1.54663743906977 3543.40111578359
1.61990800024395 3926.06239167527
1.69000487886171 4317.14333702743
1.75730789900303 4717.12879862109
1.82212667320922 5126.55018167644
1.88471753176505 5545.99185421269
1.94529553946556 5976.0987168996
2.00404323735458 6417.58521344896
2.06111713847338 6871.24612225125
2.11665264505793 7337.96958510755
2.1707678321628 7818.75295467511
2.22356640163383 8314.72224806523
2.27514001850367 8827.15625233288
2.32557018064795 9357.51670762848
2.37492973084278 9907.48654853509
2.42328409142697 10479.0189765846
2.47069228134243 11074.4013469418
2.51720776067593 11696.3396961176
2.56287913716765 12348.0726453624
2.60775076129818 13033.5281305727
2.65186323070653 13757.5443237197
2.69525382027284 14526.1899146087
2.73795685083191 15347.2441508942
2.78000400689269 16230.9456576159
2.82142461172746 17191.2192207557
2.86224586662004 18247.8134492887
2.9024930598204 19430.3365989971
2.94218974976591 20786.7549552248
2.98135792633989 22404.3765552702
3.02001815330168 24477.132776658
3.05818969450781 27686.4341351164
3.09589062612293 60604.7367382226
3.13313793667592 62267.3306556756
3.16994761653247 63651.8767800161
3.20633473812108 64854.0003360527
3.24231352805431 65925.1989624745
3.27789743212414 66896.8638024415
3.31309917401337 67789.7512506825
3.34793080845022 68618.4032399159
3.38240376943561 69393.4596711254
3.41652891409018 70122.9731701597
3.45031656259774 70813.2057695201
3.48377653466154 71469.1366375532
3.51691818283855 72094.7992028596
3.54975042307237 72693.5127477348
3.58228176270719 73268.04615849
3.61452032623231 73820.7365353523
3.64647387897796 74353.577049637
3.67814984895812 74868.2831356681
3.70955534703456 75366.343267635
3.74069718555709 75849.0584399384
3.77158189561844 76317.5732904656
3.80221574304758 76772.9009091922
3.83260474325241 77215.9428404241
3.86275467501147 77647.5053551995
3.89267109330405 78068.3128118444
3.92235934125968 78479.018719511
3.95182456129939 78880.2149507901
};
\addplot [semithick, color5, mark=*, mark size=1, mark options={solid}]
table {%
1 1268.82868775917
1.10996151344182 1575.05802024631
1.20997071148193 1886.12026197023
1.30232241935941 2202.21379773944
1.38854537026638 2523.5504741739
1.46971861477299 2850.35689401489
1.54663743906977 3182.87587513696
1.61990800024395 3521.36809814379
1.69000487886171 3866.11397301945
1.75730789900303 4217.4157647887
1.82212667320922 4575.60001730141
1.88471753176505 4941.02033443108
1.94529553946556 5314.06057831588
2.00404323735458 5695.1385694657
2.06111713847338 6084.71038352431
2.11665264505793 6483.27536643047
2.1707678321628 6891.38202520844
2.22356640163383 7309.63497755628
2.27514001850367 7738.70321414165
2.32557018064795 8179.32997495534
2.37492973084278 8632.34465296927
2.42328409142697 9098.67724862147
2.47069228134243 9579.37607223338
2.51720776067593 10075.6296358352
2.56287913716765 10588.794003145
2.60775076129818 11120.4273476226
2.65186323070653 11672.3341837221
2.69525382027284 12246.6227764755
2.73795685083191 12845.7808515076
2.78000400689269 13472.7772471925
2.82142461172746 14131.2012233633
2.86224586662004 14825.4579311768
2.9024930598204 15561.0503423298
2.94218974976591 16344.9992760867
2.98135792633989 17186.4940437196
3.02001815330168 18097.9494922264
3.05818969450781 19096.8290326469
3.09589062612293 20209.0416065191
3.13313793667592 21475.970206772
3.16994761653247 22971.3891448233
3.20633473812108 24853.4294810608
3.24231352805431 27628.5741492578
3.27789743212414 67971.5517060323
3.31309917401337 69105.1416733301
3.34793080845022 70120.0946247003
3.38240376943561 71044.1745834219
3.41652891409018 71895.906585186
3.45031656259774 72688.3618266287
3.48377653466154 73431.1596717701
3.51691818283855 74131.6157929157
3.54975042307237 74795.4428143908
3.58228176270719 75427.1994959916
3.61452032623231 76030.5904912877
3.64647387897796 76608.6731994837
3.67814984895812 77164.0045519713
3.70955534703456 77698.7477761974
3.74069718555709 78214.7516479251
3.77158189561844 78713.6104264719
3.80221574304758 79196.7099098472
3.83260474325241 79665.2633847461
3.86275467501147 80120.339956786
3.89267109330405 80562.8872403561
3.92235934125968 80993.7496498591
3.95182456129939 81413.6833305709
};
\addplot [semithick, color6, mark=*, mark size=1, mark options={solid}]
table {%
1 1153.11534113191
1.10996151344182 1430.03226695262
1.20997071148193 1710.73804990963
1.30232241935941 1995.36761972763
1.38854537026638 2284.06383596786
1.46971861477299 2576.97814784539
1.54663743906977 2874.271324898
1.61990800024395 3176.11426828301
1.69000487886171 3482.68891680214
1.75730789900303 3794.18925621961
1.82212667320922 4110.82245204988
1.88471753176505 4432.81012075285
1.94529553946556 4760.38976134927
2.00404323735458 5093.81637550133
2.06111713847338 5433.36430322558
2.11665264505793 5779.32931126422
2.1707678321628 6132.03098200405
2.22356640163383 6491.81544913815
2.27514001850367 6859.0585527477
2.32557018064795 7234.16948577099
2.37492973084278 7617.59503665301
2.42328409142697 8009.82454518515
2.47069228134243 8411.39572435789
2.51720776067593 8822.9015440492
2.56287913716765 9244.9984176772
2.60775076129818 9678.41600447315
2.65186323070653 10123.9690407346
2.69525382027284 10582.5717252035
2.73795685083191 11055.255374458
2.78000400689269 11543.1902961492
2.82142461172746 12047.7131731641
2.86224586662004 12570.3617592033
2.9024930598204 13112.9194050828
2.94218974976591 13677.4730410114
2.98135792633989 14266.4899249147
3.02001815330168 14882.92113088
3.05818969450781 15530.3440849035
3.09589062612293 16213.1637448906
3.13313793667592 16936.9048102428
3.16994761653247 17708.6507936515
3.20633473812108 18537.731333299
3.24231352805431 19436.8536726255
3.27789743212414 20424.0877878307
3.31309917401337 21526.6517346132
3.34793080845022 22789.0048056135
3.38240376943561 24293.3410848102
3.41652891409018 26228.6066943418
3.45031656259774 29348.8257048927
3.48377653466154 75171.1397221552
3.51691818283855 75984.6568808932
3.54975042307237 76743.4666071406
3.58228176270719 77456.2449920176
3.61452032623231 78129.6279619962
3.64647387897796 78768.8222956222
3.67814984895812 79377.9997469359
3.70955534703456 79960.5631547749
3.74069718555709 80519.3292809379
3.77158189561844 81056.6605048151
3.80221574304758 81574.5604661358
3.83260474325241 82074.7454388648
3.86275467501147 82558.6984882122
3.89267109330405 83027.7111994293
3.92235934125968 83482.9163354476
3.95182456129939 83925.3136700693
};
\addplot [semithick, gray!99.6078431372549!black, mark=*, mark size=1, mark options={solid}]
table {%
1 1051.90116643941
1.10996151344182 1303.44726288011
1.20997071148193 1557.99394172329
1.30232241935941 1815.63441347309
1.38854537026638 2076.46666063304
1.46971861477299 2340.59378187961
1.54663743906977 2608.12436994593
1.61990800024395 2879.17292492343
1.69000487886171 3153.86030778706
1.75730789900303 3432.31424093356
1.82212667320922 3714.66985891778
1.88471753176505 4001.07031664975
1.94529553946556 4291.66746463481
2.00404323735458 4586.622597123
2.06111713847338 4886.1072842512
2.11665264505793 5190.30430224765
2.1707678321628 5499.40867237775
2.22356640163383 5813.62882592496
2.27514001850367 6133.18791706005
2.32557018064795 6458.32530267821
2.37492973084278 6789.29821739155
2.42328409142697 7126.38367974316
2.47069228134243 7469.88066200891
2.51720776067593 7820.1125779211
2.56287913716765 8177.43014050562
2.60775076129818 8542.21466092436
2.65186323070653 8914.88187908926
2.69525382027284 9295.88642441276
2.73795685083191 9685.72704972988
2.78000400689269 10084.9527957754
2.82142461172746 10494.1703087648
2.86224586662004 10914.0525718883
2.9024930598204 11345.3494129332
2.94218974976591 11788.9002336084
2.98135792633989 12245.6495758424
3.02001815330168 12716.6663249425
3.05818969450781 13203.1676488722
3.09589062612293 13706.549171899
3.13313793667592 14228.4234937297
3.16994761653247 14770.6700428606
3.20633473812108 15335.5006187596
3.24231352805431 15925.5470950771
3.27789743212414 16543.9811598262
3.31309917401337 17194.6816380418
3.34793080845022 17882.4746784895
3.38240376943561 18613.4896944253
3.41652891409018 19395.7073085347
3.45031656259774 20239.8430326763
3.48377653466154 21160.8577524495
3.51691818283855 22180.7408622233
3.54975042307237 23334.1838011124
3.58228176270719 24681.9491843409
3.61452032623231 26350.5329837158
3.64647387897796 28717.9644245379
3.67814984895812 81469.8024553932
3.70955534703456 82117.9322483101
3.74069718555709 82734.115466073
3.77158189561844 83322.1707454383
3.80221574304758 83885.2242523683
3.83260474325241 84425.872718139
3.86275467501147 84946.3003285067
3.89267109330405 85448.3644825201
3.92235934125968 85933.6599597318
3.95182456129939 86403.5678020375
};
\end{axis}

\end{tikzpicture}
	\caption{}
%	\label{fig:bist_test}
\end{figure}

\begin{figure}[ht]
	\centering
	\tikzsetexternalprefix{tikz/}	% set subfolder
\tikzsetnextfilename{bistability_test2}
% This file was created by matplotlib2tikz v0.6.15.

\newcommand{\downfile}[1]{
    \pgfplotstableread[col sep=tab, header=true]{#1}{\table}
    \pgfplotstablegetcolsof{#1}
    \pgfmathtruncatemacro\numberofcols{\pgfplotsretval - 1}
    \pgfplotsinvokeforeach{1,...,\numberofcols}{
        \pgfplotstablegetcolumnnamebyindex{##1}\of{\table}\to{\colname}
        \addplot [name path=down##1, semithick, color##1, mark=*, mark size=1, mark options={solid}]%
        		table [x index= 0, y index=##1] {#1};
        \addlegendentryexpanded{ \colname }
    }
}

\newcommand{\upfile}[1]{
    \pgfplotstableread[col sep=tab, header=true]{#1}{\table}
    \pgfplotstablegetcolsof{#1}
    \pgfmathtruncatemacro\numberofcols{\pgfplotsretval - 1}
    \pgfplotsinvokeforeach{1,...,\numberofcols}{
        \pgfplotstablegetcolumnnamebyindex{##1}\of{\table}\to{\colname}
        \addplot [name path=up##1, semithick, color##1, mark=*, mark size=1, mark options={solid}]%
        		table [x index= 0, y index=##1] {#1};
    }
}

\begin{tikzpicture}[baseline]

	\definecolor{color1}{rgb}{0.12156862745098,0.466666666666667,0.705882352941177}
	\definecolor{color2}{rgb}{1,0.498039215686275,0.0549019607843137}
	\definecolor{color3}{rgb}{0.172549019607843,0.627450980392157,0.172549019607843}
	\definecolor{color4}{rgb}{0.83921568627451,0.152941176470588,0.156862745098039}
	\definecolor{color5}{rgb}{0.580392156862745,0.403921568627451,0.741176470588235}
	\definecolor{color6}{rgb}{0.549019607843137,0.337254901960784,0.294117647058824}
	\definecolor{color7}{rgb}{0.890196078431372,0.466666666666667,0.76078431372549}
	\definecolor{color8}{rgb}{0.45,0.45,0.45}
	\definecolor{color9}{rgb}{0.25,0.25,0.99}
	
	\begin{axis}[
			title={Internal Power},
			xlabel={Pump Power [\si{mW}]},
			ylabel={Internal Power [arb.units]},
			tick align=outside,
			tick pos=left,
			width=\textwidth*0.75,%
			height=207pt,
			legend pos = outer north east,
			cycle list name=color list,
			forget plot style={opacity=0.4},
		]
		\addlegendentry{\hspace{-.6cm}$\Delta\lambda$ in \si{\pm}}
		\addlegendimage{empty legend};
		
		\upfile{tikz/upwards.csv}
		\downfile{tikz/downwards.csv}

    \addplot [fill=color1, opacity=0.2] fill between [of=up1 and down1];
    \addplot [fill=color2, opacity=0.2] fill between [of=up2 and down2];
    \addplot [fill=color3, opacity=0.2] fill between [of=up3 and down3];
    \addplot [fill=color4, opacity=0.2] fill between [of=up4 and down4];
    \addplot [fill=color5, opacity=0.2] fill between [of=up5 and down5];
    \addplot [fill=color6, opacity=0.2] fill between [of=up6 and down6];
    \addplot [fill=color7, opacity=0.2] fill between [of=up7 and down7];
    \addplot [fill=color8, opacity=0.2] fill between [of=up8 and down8];
    \addplot [fill=color9, opacity=0.2] fill between [of=up9 and down9];
	\end{axis}
\end{tikzpicture}
	\caption{}
%	\label{fig:bist_test}
\end{figure}