\documentclass[11pt]{beamer}
%\usetheme{Dresden}%{Berkeley}
\usetheme{metropolis} 
\usepackage[utf8]{inputenc}
\usepackage[english]{babel}
\usepackage{amsmath}
\usepackage{amsfonts}
\usepackage{amssymb}
\usepackage{graphicx}
\usepackage{wrapfig}
\usepackage[font=scriptsize,labelfont=bf]{subcaption}
\usepackage[font=scriptsize,labelfont=bf]{caption}
\usepackage{FiraSans}
\usepackage{FiraMono}

\usepackage{lmodern}

\usepackage[binary-units = true]{siunitx}
\DeclareSIUnit{\bps}{bps}

% create plots
%\usepackage{pgfplots} % README here http://pgfplots.sourceforge.net/pgfplots.pdf
%\pgfplotsset{compat=1.14}

% externalize tikz images
\usepackage{tikz}							% externalize tikz images
\usepackage{pgfplots} 					% create plots with pgf/tikz
\pgfplotsset{compat=1.14} 			% README here http://pgfplots.sourceforge.net/pgfplots.pdf
	%\usepgfplotslibrary{external}
	\usetikzlibrary{external}			%
	
	\makeatletter
	\newcommand*{\overlaynumber}{\number\beamer@slideinframe}
	\tikzset{
	  beamer externalizing/.style={%
	    execute at end picture={%
	      \tikzifexternalizing{%
	        \ifbeamer@anotherslide
	        \pgfexternalstorecommand{\string\global\string\beamer@anotherslidetrue}%
	        \fi
	      }{}%
	    }%
	  },
	  external/optimize=false
	}
	\let\orig@tikzsetnextfilename=\tikzsetnextfilename
	\renewcommand\tikzsetnextfilename[1]{\orig@tikzsetnextfilename{#1-\overlaynumber}}
	\makeatother
	
	\tikzset{every picture/.style={beamer externalizing}}	
	
	\tikzexternalize								% activate!
	\tikzsetexternalprefix{tikz/}	% set subfolder
	
\usepgfplotslibrary{fillbetween,dateplot} % need this to fill between functions
\usetikzlibrary{	patterns, decorations, decorations.markings, decorations.pathreplacing,
								shapes, shapes.geometric, shapes.misc, arrows, arrows.meta,
								positioning, intersections,
								overlay-beamer-styles,
								mindmap,trees,shadows}

% should do transition things
%\tikzset{
%    invisible/.style={opacity=0,text opacity=0},
%    visible on/.style={alt=#1{}{invisible}},
%    alt/.code args={<#1>#2#3}{%
%      \alt<#1>{\pgfkeysalso{#2}}{\pgfkeysalso{#3}} % \pgfkeysalso doesn't change the path
%    },
%}

\usepackage[
  backend     = biber,
  style       = phys,
  autocite    = superscript,
  sortcites   = true,
]{biblatex}
\addbibresource{../tex/bib_articles.bib}
\addbibresource{../tex/bib_books.bib}
\addbibresource{../tex/bib_websites.bib}
\usepackage{csquotes}
\renewcommand{\footnotesize}{\tiny}%{\scriptsize}
%\renewcommand*{\bibfont}{\scriptsize}
%\setbeamertemplate{bibliography item}{\insertbiblabel}

\author[D. Bazzanella]{Davide Bazzanella}
\title[All-Optical Neural Networks]{Optical Bistability As Neural Network Nonlinear Activation Function}
%\subtitle[short subtitle]{long subtitle}
%\logo{\includegraphics[height=2.3cm]{pics/unitn_logo.png} } 
\institute[UNITN]{Università degli studi di Trento} 
\date{20th March 2018} 
\subject{Master Thesis in Physics}

\setbeamercovered{transparent} 
\setbeamertemplate{navigation symbols}{} 

\newcommand{\frameofframes}{/}
\newcommand{\setframeofframes}[1]{\renewcommand{\frameofframes}{#1}}
\setframeofframes{of}

\setbeamertemplate{footline}{
	\begin{beamercolorbox}[ht=2.5ex, dp=1.125ex, leftskip=.3cm, rightskip=.3cm plus1fil]
		{title in head/foot}
		{\insertshorttitle} \hfill {\insertframenumber~\frameofframes~\inserttotalframenumber}
%		{author in head/foot}
%		{\insertshortauthor} \hfill {\insertinstitute}
	\end{beamercolorbox}
}

% set automatic fade transition
\addtobeamertemplate{background canvas}{\transfade}{}

% \section[Text]{Long Text}: Long Text is used in TOC, Text in navigation.
% \footfullcite{ <citation> }

\begin{document}

\frame{\titlepage}
%%%%% %%%%% %%%%% %%%%% %%%%% %%%%%
\section[Introduction]{Introduction}
\begin{frame}{All-optical Artificial Neural Networks}
	Applying integrated photonics to artificial neural networks
\end{frame}

%%%%% %%%%% %%%%% %%%%% %%%%% %%%%%
\begin{frame}{Outline}
\tableofcontents[pausesections]
\end{frame}

%%%%% %%%%% %%%%% %%%%% %%%%% %%%%%
\section{Artificial Neural Networks}
%
\begin{frame}{ANNs}
	Artificial Neural Networks are computation systems, composed by a collection of nodes that work seemingly biological neurons.
\end{frame}
%
\begin{frame}{ANNs blocks}
	ANNs are composed by single units, \textit{nodes}, which elaborate the information in a way loosely similar to biological neurons.
	\begin{figure}
		\centering
		\tikzsetexternalprefix{tikz/}	% set subfolder
\tikzsetnextfilename{node}
\begin{tikzpicture}[baseline, scale=1]
	\node [shape=circle, minimum size=2cm, draw=gray!100, fill=gray!20]%
				(af) at (1.5,0) {\LARGE $f_a$};
	\node (aftext) [below, align=center] at (af.south) {activation \\ function};
	\node [regular polygon, regular polygon sides=6, minimum size=2cm, draw=gray!100, fill=gray!20]%
				(ws) at (-2,0) {\LARGE $\Sigma$};
	\draw (aftext.north)++(-3.5,0) node (wstext) [below, align=center] {weighted \\ sum};
	\draw (ws) to (af);
	\foreach \i in {1,2,4}	\draw (-4.5,1.5-0.6*\i) .. controls (-3.5,1.5-0.6*\i) .. (ws);
	\foreach \i in {1,2}		\node at (-4.5,1.5-0.6*\i) [left] {$x_\i$};
	\node (xn) at (-4.5,-0.9)  [left] {$x_n$};
	\draw [dashed] (-4.5,-0.3) .. controls (-3.5,-0.3) .. (ws);
	\node at (-4.5, -0.2) {$\vdots\qquad$};
	\draw (-2,1.5) node [above] {$w_0$} to (ws);
	\draw (af) to (3.5,0) node [right] {$y$};
\end{tikzpicture}
	\end{figure}
\end{frame}
%
\begin{frame}[c]{What can they do?}
	\center \huge What can they do?
\end{frame}
%
\begin{frame}{What can they do?}
	\begin{columns}
		\column{0.6\textwidth}
			ANNs can solve complex problems:
			\begin{itemize}
				\item \alert<2>{\textbf<2>{classification}}
				\item \alert<3>{\textbf<3>{clustering}}
				\item \alert<4>{\textbf<4>{pattern recognition}}
				\item \alert<5>{\textbf<5>{time series prediction}}
			\end{itemize}
		\column{0.35\textwidth}
			\begin{figure}
				\centering
				\only<2>{\includegraphics[draft,width=3cm,height=2cm]{figures/foo.png}}
				\only<3>{\includegraphics[draft,width=3cm,height=3cm]{figures/foo.png}}
				\only<4>{\includegraphics[draft,width=3cm,height=4cm]{figures/foo.png}}
				\only<5>{\includegraphics[draft,width=3cm,height=5cm]{figures/foo.png}}
			\end{figure}
	\end{columns}
\end{frame}
%
\begin{frame}[c]{How do they work?}
	\center \huge How do they work?
	
	\tikzsetexternalprefix{tikz/}	% set subfolder
	\tikzsetnextfilename{test}
	\tikzset{external/export next=false}
	\begin{tikzpicture}[remember picture,overlay]
    \draw [line width=10mm,opacity=.25] (current page.center) circle (3cm);
		%\node [rotate=60,scale=2,text opacity=0.2] at (current page.center) {Example};
	\end{tikzpicture}
\end{frame}
\begin{frame}{How do they work?}
	ANNs can obtain arbitrary decision regions\footnotemark
	
	\vspace*{2em}
	The amount of free parameters in an ANN, allow ..?
	\footnotetext{\fullcite{duda2012pattern}}%\footfullcite{duda2012pattern}
\end{frame}
\begin{frame}{How do they work?}
	\begin{itemize}[<+->]
		\item training
		\begin{itemize}
			\item evaluate loss
			\item adjust parameters
		\end{itemize}
		\item validation
		\item test
	\end{itemize}
\end{frame}
%

%%%%% %%%%% %%%%% %%%%% %%%%% %%%%%
\section{Microring Resonator}
\begin{frame}{MRR}
	\begin{columns}
		\column{0.45\textwidth}
		placeholder
		\column{0.45\textwidth}
		\begin{figure}
			\centering
			\tikzsetexternalprefix{tikz/}	% set subfolder
\tikzsetnextfilename{MRR}
\begin{tikzpicture}[
		baseline,
		scale=0.3,
		every pin edge/.style={-},
	]

%	\draw [help lines] (-3,-3) grid (3,3);
	
	\filldraw [draw=black!50, fill=gray!20, even odd rule]
		(0,0) circle [radius=4.43, thick]
		(0,0) circle [radius=4.91, thick];
	\path [name path=inner] (0,0) circle [radius=5.085];
	\path [name path=outer] (0,0) circle [radius=5.505];
%	%distance between waveguides is 9.848
	\path [name path=belowwg] (-6,-4.924) rectangle (+6,-5.344);
	\path [name path=abovewg] (-6,+4.924) rectangle (+6,+5.344);
	
	\path [name intersections={of=inner and abovewg, name=abovein}];
	\path [name intersections={of=outer and abovewg, name=aboveout}];
	
	\path [name intersections={of=inner and belowwg, name=belowin}];
	\path [name intersections={of=outer and belowwg, name=belowout}];
	
	\draw [black!50, line join=round, rounded corners=1pt, fill=gray!20]
		(-6,+4.924) -- (-6,+5.344) -- (aboveout-3)
		.. controls ++(+13:1) and ++(+167:1) .. (aboveout-1)
		-- (+6,+5.344) -- (+6,+4.924) -- (abovein-1)
		.. controls ++(+168:1) and ++(+12:1) .. (abovein-2)
		-- cycle;
	
	\draw [black!50, line join=round, rounded corners=1pt, fill=gray!20]
		(-6,-4.924) -- (-6,-5.344) -- (belowout-1)
		.. controls ++(-13:1) and ++(+193:1) .. (belowout-3)
		-- (+6,-5.344) -- (+6,-4.924) -- (belowin-2)
		.. controls ++(+192:1) and ++(-12:1) .. (belowin-1)
		-- cycle;
	
	\draw [semithick, -stealth] (-6,+6.0) -- (-5,+6.0) node [midway, above] {\texttt{I}};
	\draw [semithick, -stealth] (+5,+6.0) -- (+6,+6.0) node [midway, above] {\texttt{T}};
	\draw [semithick, -stealth] (-5,-6.0) -- (-6,-6.0) node [midway, below] {\texttt{D}};
	\draw [semithick, -stealth] (+6,-6.0) -- (+5,-6.0) node [midway, below] {\texttt{A}};
	
	\draw [-stealth, visible on=<1>]  (0,0) -- ++(60:4.43) node [left, midway] {\scriptsize $R_{in}$};
	\draw [-stealth, visible on=<2->] (0,0) -- ++(60:4.43) node [left, midway] {\scriptsize\SI{4.43}{\um}};
				
	\draw [|-|, black, visible on=<2->] (+4.430,+0) -- (+4.910,+0) node [pin=+45:\scriptsize\SI{0.48 }{\um}] {};
	\draw [|-|, black, visible on=<2->] (+5,-5.344) -- (+5,-4.924) node [pin=+45:\scriptsize\SI{0.42 }{\um}] {};
	\draw [|-|, black, visible on=<2->] (+0,-4.924) -- (+0,-5.085) node [pin=+90:\scriptsize\SI{0.175}{\um}] {};

\end{tikzpicture}
%			\includegraphics[draft,width=4.5cm,height=3cm]{figures/foo.png}
		\end{figure}
	\end{columns}
\end{frame}
\begin{frame}{Theory}
Linear
\end{frame}
\begin{frame}{Theory}
Nonlinear
\end{frame}
\begin{frame}{Experiments}
Setup
\end{frame}

%%%%% %%%%% %%%%% %%%%% %%%%% %%%%%
\section{ANN Simulations}
\begin{frame}{Simulation Framework}
What means simulating?
PyTorch library
\end{frame}
\begin{frame}{Fundamental blocks}
model ($FF[f_a]$)\\
loss criteria (CEL)\\
weight update criteria (SGD)
\end{frame}
\begin{frame}{Operation Tests}
ReLU vs Sigmoid vs $f_{fit}$
\end{frame}

%%%%% %%%%% %%%%% %%%%% %%%%% %%%%%
\section{Conclusion}
\begin{frame}{Conclusions}
	Overview
	Improvements
	Future Perspective
\end{frame}

%\section[]{References}
%\begin{frame}[allowframebreaks]{References}
%\printbibliography
%\end{frame}

\begin{frame}{Mindmap}
\tikzsetexternalprefix{tikz/}	% set subfolder
\tikzsetnextfilename{mindmap}
\begin{tikzpicture}[mindmap, concept color=gray!50!violet, font=\sf, text=white]

  \tikzstyle{level 1 concept}+=[font=\sf, sibling angle=90,level distance = 30mm]

  \node[concept,scale=0.7] {Center}
    [clockwise from={90+45}]
        child[concept color=red, visible on=<2->]{ node[concept,scale=0.7]{NW} } 
        child[concept color=orange, visible on=<3->]{ node[concept,scale=0.7]{NE} } 
        child[concept color=yellow, visible on=<4->]{
        		node[concept,scale=0.7]{SE}
%        			[clockwise from={45}]
%        				child[concept color=yellow!50!orange, scale=0.3, visible on=<5->] {SE-to-NE}
%        				child[concept color=yellow!, scale=0.3, visible on=<7->] {SE-to-SE}
%        				child[concept color=yellow!50!green, scale=0.3, visible on=<6->] {SE-to-SW}
        		} 
        child[concept color=green, visible on=<5->]{ node[concept,scale=0.7]{SW} };

\end{tikzpicture}
\end{frame}

\end{document}

%\begin{itemize}
% \item<1-> Text visible on slide 1
% \item<2-> Text visible on slide 2
% \item<3> Text visible on slide 3
% \item<4-> Text visible on slide 4
%\end{itemize}