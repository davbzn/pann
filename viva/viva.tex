\documentclass[11pt]{beamer}
%\usetheme{Dresden}%{Berkeley}
\usetheme{metropolis} 
\usepackage[utf8]{inputenc}
\usepackage[english]{babel}
\usepackage{amsmath}
\usepackage{amsfonts}
\usepackage{amssymb}
\usepackage{graphicx}
\usepackage[font=scriptsize,labelfont=bf]{subcaption}
\usepackage[font=scriptsize,labelfont=bf]{caption}
%\usepackage{FiraSans}
%\usepackage{FiraMono}

\usepackage{lmodern}

\usepackage[binary-units = true]{siunitx}
\DeclareSIUnit{\bps}{bps}

% create plots
%\usepackage{pgfplots} % README here http://pgfplots.sourceforge.net/pgfplots.pdf
%\pgfplotsset{compat=1.14}

% externalize tikz images
\usepackage{tikz}\usetikzlibrary{external}
\tikzexternalize[prefix=tikz/] % activate!

%\usepackage[
%  backend     = biber,
%  style       = phys,
%  autocite    = superscript,
%  sortcites   = true,
%]{biblatex}
%\bibliography{optical_buffers}
%\usepackage{csquotes}
%\renewcommand*{\bibfont}{\scriptsize}
%\setbeamertemplate{bibliography item}{\insertbiblabel}

\author[D. Bazzanella]{Davide Bazzanella}
\title[All-Optical Neural Networks]{Optical Bistability As Neural Network Nonlinear Activation Function}
%\logo{\includegraphics[height=2.3cm]{pics/unitn_logo.png} } 
\institute[UNITN]{Università degli studi di Trento} 
\date{20 March 2018} 
\subject{Master Thesis in Physics}

\setbeamercovered{transparent} 
\setbeamertemplate{navigation symbols}{} 

\newcommand{\frameofframes}{/}
\newcommand{\setframeofframes}[1]{\renewcommand{\frameofframes}{#1}}
\setframeofframes{of}

\setbeamertemplate{footline}{
	\begin{beamercolorbox}[ht=2.5ex, dp=1.125ex, leftskip=.3cm, rightskip=.3cm plus1fil]
		{title in head/foot}
		{\insertshorttitle} \hfill {\insertframenumber~\frameofframes~\inserttotalframenumber}
%		{author in head/foot}
%		{\insertshortauthor} \hfill {\insertinstitute}
	\end{beamercolorbox}
}

\begin{document}

\frame{\titlepage}

\begin{frame}{Outline}
\tableofcontents
\end{frame}

\section{Introduction}
\begin{frame}{What is an optical buffer?}
	In telecommunications and computing, a buffer is a \textbf{storage medium} that keeps data for a certain time.
	Data are usually stored in a \textit{very temporary} way.
	
	\vspace{1em}
	An optical buffer is a device that is capable of temporarily storing data that was \textbf{transmitted optically}, without converting it to the electrical domain.
\end{frame}

%\subsubsection*{Fields of study}
%\begin{frame}{Fields of interest and Purpose}
%%	\begin{columns}
%%		\column{0.55\textwidth}
%			\textbf{Fields of interest}:\autocite{Tanabe2007,Song2005,Baba2008,Tucker2005}
%			\begin{itemize}
%				\item optical signal processing,
%				\item optical communications,
%				\item signal production and synchronization
%			\end{itemize}
%		
%			\vspace{1em}
%			\textbf{Purpose}:\autocite{Tanabe2007,Song2005,Baba2008,Tucker2005}
%			\begin{itemize}
%				\item Substitute electronic buffers,
%				\item chip integration of bulky delay lines
%			\end{itemize}
%		
%%		\column{0.4\textwidth}
%%%			\only<2>
%%			{
%%				placeholder
%%			}
%%	\end{columns}
%\end{frame}
%
%\subsubsection*{Competing technologies}
%\begin{frame}{Competing technologies}
%	\textbf{Silicon-based electronics (MOS integrated circuits)}:\\
%	\begin{itemize}
%		\item 	single bit density ($\sim\SI{100}{\square\nm}$)\autocite{Haraszti2000}
%		\item highly refined production process:\\
%			high speed, low consumption, low cost
%		\item could be difficult to improve further\autocite{skotnicki2005end}
%		\item requires conversion of domains (optical/electrical)\\ twice at switch nodes
%					%requires double conversion from optical\\ to electrical domain and vice versa
%	\end{itemize}
%\end{frame}
%
%\section{Optical delay line}
%\begin{frame}{Optical delay line}
%	\textbf{Packets} (consecutive bits) \textbf{are delayed} of a certain time $T_S$.\\
%	The delay is \textbf{controlled by an external signal}.
%	\vspace{1em}
%	\begin{figure}[hb]
%		\centering
%		\includegraphics[width=\textwidth]{pics/OB_memory_edited.png}%scale=.11
%		\caption{Physical sizes are retrieved by $L=v_g\tau$. Adapted from \autocite{Tucker2005}}
%		\label{fig:OB_memory}
%	\end{figure}
%\end{frame}
%
%\subsubsection*{Key parameters}
%\begin{frame}{Key parameters}
%	Key operating parameters are
%	\begin{itemize}
%		\item Storage time $T_S$
%		\item	Bit period $\tau_b$
%		\item Delay-bandwidth Product (DBP), i.e. the capacity $C=T_S /\tau_b$
%	\end{itemize}
%	
%	\vspace{0.5em} Important physical traits are
%	\begin{itemize}
%		\item bit size
%		\item intrinsic and extrinsic losses
%	\end{itemize}
%	
%	\vspace{0.5em} Other parameters are the signal wavelength $\lambda$,\\ the Hold-off time $T_{HO}$ and the slow down factor $S=v_{g1}/v_{g2}$.
%\end{frame}
%
%\subsection{Implementation}
%\begin{frame}{Implementation}
%
%	Considering a FIFO buffer for simplicity, although parallel operation is also possible.
%
%	\vspace{.5em}
%	Implementation:
%	\begin{itemize}
%		\item long optical path
%		\item optical cavity: high $Q$ factor and long photon lifetime
%		\item slow propagation speed: \textit{slow light}
%	\end{itemize}
%
%	\vspace{.5em}
%	Also bear in mind that:	
%	\begin{quote}
%		A delay line whose delay cannot be changed is not an optical buffer.\autocite{Tucker2005}
%	\end{quote}
%\end{frame}
%
%\subsubsection*{Optical cavity}
%\begin{frame}{Optical cavity}
%	\begin{columns}
%		\column{0.54\textwidth}			
%			Integrated cavities are obtained with simple waveguides (\textbf{rings}) and with more complex \textbf{PhC lattices} and defects.
%			
%			\vspace{0.5em} PhCs can employ semiconductor \textbf{heterojunctions} design techniques to boost their quality parameters.\autocite{Istrate2006}
%			
%			\vspace{0.5em} Control over $v_g$ is obtained mainly by changing the input/output coupling of the cavity.\autocite{Tucker2005}
%
%		\column{0.45\textwidth}
%			\vspace{-1em}
%			\begin{figure}
%		    \begin{subfigure}[b]{\textwidth}
%					\centering
%					\includegraphics[width=.7\textwidth]{pics/CROW_edited.png}
%					\caption{Rings. Adapted from \autocite{Xia2007a}}
%					\label{fig:CROW}
%				\end{subfigure}
%		    \begin{subfigure}[b]{\textwidth}
%					\centering
%					\includegraphics[width=.65\textwidth]{pics/HJ_cavity_edited.png}
%					\caption{Heterojunction. Adapted from \autocite{Song2005}}
%					\label{fig:PhC_cavity}
%				\end{subfigure}
%				\label{figs:CROW_and_PhC_cavity}
%			\end{figure}
%	\end{columns}
%\end{frame}
%
%\subsubsection*{Slow light}
%\begin{frame}{Slow light}
%	Slow light is a phenomenon in which the propagation of light in a medium is reduced greatly thanks to the \textbf{low group velocity} $\frac{\partial \omega}{\partial k}$.
%	
%	\vspace{1em}
%	To obtain such effect one has to employ either:
%	\begin{itemize}
%		\item Near resonance refractive index\autocite{yu2013slow}
%		\item Electromagnetically Induced Transparency (EIT)
%		\item Population Oscillations (PO)
%		\item Dispersion band-edge in Photonic Crystals (PhC)
%	\end{itemize}
%	
%	\vspace{0.5em} Control over $v_g$ is obtained in EIT devices by changing the wavelength and power of the pump laser.\autocite{chang2003variable}
%\end{frame}
%
%\subsection{Slow light DL optical buffers}
%\begin{frame}{Types of delay line buffers}
%	Focusing on the delay line optical buffer based on slow light, we can separate them in two categories:
%	\begin{itemize}
%	\item A-type: cannot change its $v_g$ when storing a bit
%	\item B-type: can change its $v_g$ when storing a bit
%	\end{itemize}
%\end{frame}
%
%\begin{frame}{A-type}
%	A-type: cannot change its $v_g$ when storing a bit
%	\begin{figure}[hb]
%		\centering
%		\includegraphics[width=.65\textwidth]{pics/CLA_pulses_edited.png}
%		\caption{Adapted from \autocite{Tucker2005}}
%		\label{fig:A_DL}
%	\end{figure}
%\end{frame}
%
%\begin{frame}{B-type}
%	B-type: can change its $v_g$ when storing a bit
%	\begin{figure}[hb]
%		\centering
%		\includegraphics[width=.8\textwidth]{pics/CLB_pulses_edited.png}
%		\caption{Adapted from \autocite{Tucker2005}}
%		\label{fig:B_DL}
%	\end{figure}
%\end{frame}
%
%\begin{frame}{Delay line composition}
%	\begin{itemize}
%		\item A-type or B-type only
%		\item A/B/A composition:
%	\end{itemize}
%		\begin{figure}[hb]
%			\centering
%			\includegraphics[width=.8\textwidth]{pics/ABA_DL.png}
%			\caption{Adapted from \autocite{Tucker2005}}
%			\label{fig:ABA_DL}
%	\end{figure}
%\end{frame}
%
%\subsubsection*{Strengths and weaknesses}
%\begin{frame}{Strengths and weaknesses}
%Bit size ranges from sub wavelength (\SI{740}{\nm}\autocite{Tucker2005}) for A-type buffers to a few \SI{10}{\um}\autocite{ku2004slow} or even more\autocite{liu2001observation} for B devices.
%On the other hand, DBP is much higher (e.g. $10^4$)\autocite{yanik2004stopping} in B-type devices.
%
%\vspace{0.5em} The hybrid A/B/A device can (in theory) get the best from the two worlds\autocite{Tucker2005}.
%
%\vspace{0.5em} The delay of slow light is dominated by \textbf{intrinsic and extrinsic losses} and dispersive components (mainly GVD), hence dispersion-compensated and zero-dispersion slow light are very important.\autocite{Baba2008}
%
%\vspace{0.5em} Pulse distortion distortion might also be too high for certain purposes.
%\end{frame}
%
%\section{Bistable optical devices}
%\begin{frame}{Bistable optical devices}
%	\begin{columns}
%		\column{0.5\textwidth}
%			An ideal bistable system has\\ \textbf{two stable output values}.\\
%			The system is forced to \textbf{switch} from one state to the other by a \textbf{control parameter}. \autocite{saleh1991fundamentals}%[\ppno 843--847]{}
%			
%			\vspace{0.5em}
%			Optical bistable systems can be used to store data in the optical domain.
%		\column{0.45\textwidth}
%			\begin{figure}[hb]
%				\centering
%				\includegraphics[width=\textwidth]{pics/bistability.png}
%				\caption{States of the polarization bistable VCSEL. \autocite{Katayama2009}}
%				\label{fig:opt_bistability}
%			\end{figure}
%	\end{columns}
%\end{frame}
%
%\subsection{Implementation}
%\begin{frame}{Implementation}
%	VCSEL provides two orthogonally polarized lasing modes with the same optical gain.
%	
%	\begin{figure}[hb]
%		\centering
%		\includegraphics[width=.75\textwidth]{pics/VCSEL_bsy_edited_v2.png}
%		\caption{All-optical flip-flop op. using polarization bistable VCSEL. Adapted from \autocite{Katayama2009}}
%		\label{fig:VCSEL_bistability}
%	\end{figure}
%\end{frame}
%
%\begin{frame}{Data, set, and reset signals}
%	Polarization bistable VCSELs act as AND gates (1-bit memories) in all-optical buffer memories.\autocite{Sakaguchi2010}
%\begin{columns}
%	\column{.7\textwidth}
%		\begin{figure}[hb]
%			\centering
%			\includegraphics[width=.9\textwidth]{pics/1-bit_OB_scheme_edited.png}
%			\caption{One-bit optical buffer memory scheme.\autocite{Katayama2009}}
%			\label{fig:1-bit_OB_scheme}
%		\end{figure}
%	\column{.3\textwidth}
%		{\small		
%		\begin{tabular}{c|c|c||c}
%			$\mathcal{I}$ & $\mathcal{S}$ & $\mathcal{R}$ & $\mathcal{O}$ \\ 
%			\hline 
%			$0$ & $0$ & $0$ & $0$ \\
%			$1$ & $0$ & $0$ & $0$ \\
%			$0$ & $1$ & $0$ & $0$ \\
%			$1$ & $1$ & $0$ & $1$ \\
%			$x$ & $y$ & $1$ & $0$ \\
%		\end{tabular} 
%		}
%\end{columns}
%	Injection light from \textbf{data} ($\ang{0}$) + \textbf{set} ($\ang{0}$) signals must be higher than the \textbf{switching threshold}.\autocite{Katayama2009}\\
%	Reset ($\ang{90}$) signal must switch back to polarization state \ang{90}.
%\end{frame}
%
%\begin{frame}
%	\begin{figure}[hb]
%		\centering
%		\includegraphics[width=.75\textwidth]{pics/1-bit_BOB_at_1gbps.png}
%		\caption{1-bit optical buffer memory operation exp. result for a \SI{1}{\giga\bps} RZ signal.\autocite{Katayama2009}}
%		\label{fig:1-bit_BOB_at_1gbps}
%	\end{figure}
%\end{frame}
%
%\subsection{Bistable optical buffers}
%\begin{frame}{Examples of bistable optical buffers}
%	\begin{figure}[hb]
%		\centering
%		\includegraphics[width=.8\textwidth]{pics/4-bit_BOB_scheme.png}
%		\caption{4-bit optical buffer memory with parallel shift register.\autocite{Katayama2009}}
%		\label{fig:4-bit_BOB}
%	\end{figure}
%	
%	4-bit buffering operation was obtained in a parallel shift register up to \SI{500}{\mega\bps}.\autocite{Katayama2009}
%\end{frame}
%
%\subsubsection*{Strengths and weaknesses}
%\begin{frame}{Strengths and weaknesses}
%	\textbf{operational records}:
%	\begin{itemize}
%		\item Buffering operation up to \SI{10}{\giga\bps} (RZ) signals\autocite{Katayama:08},
%		\item lowest switching time of of \SI{7}{\ps}\autocite{kawaguchi1997bistable},
%		\item lowest switching energy of \SI{0.3}{\femto\J}\autocite{mori2006low},
%	\end{itemize}
%	mainly using an InGaAs/GaAs VCSEL in the \SI{980}{\nm} range.
%	
%	\vspace{0.5em}
%	\textbf{operating conditions}:
%	\begin{itemize}
%		\item supply \si{\mA} bias current of the VCSEL,
%		\item 	avoid error caused by high energy data,
%		\item avoid error caused from long packages - pseudo-random bit sequence (PRBS) test
%	\end{itemize}
%\end{frame}
%
%\section{Conclusion}
%\begin{frame}{Summary}
%Optical delay lines are not adequate for optical packet switches (OPS) due to their \textbf{fundamental limits}, e.g bit size and DBP.
%However there are other application such as integration of bulky delay lines which could benefit from such devices
%
%\vspace{1em} Optical bistable VCSELs have demonstrated \textbf{high bit rates} and \textbf{low bit size}, hence they seem an interesting technology for all-optical buffering.
%On the other hand, from the works considered one cannot understand (and thus judge) the \textit{wall-plug efficiency} of the device which could be a critical factor.
%\end{frame}
%
%\begin{frame}[allowframebreaks]{References}
%\printbibliography
%\end{frame}

\end{document}

%\begin{itemize}
% \item<1-> Text visible on slide 1
% \item<2-> Text visible on slide 2
% \item<3> Text visible on slide 3
% \item<4-> Text visible on slide 4
%\end{itemize}


%\begin{frame}
%Photonic-crystal devices are especially attractive for generating slow light, as they are compatible with on-chip integration and room- temperature operation. \autocite{Baba2008}\\
%slow light (offers the opportunity for compressing optical signals and optical energy in space, which) reduces the device footprint and enhances light–matter interactions. \autocite{Baba2008}\\
%The delay of slow light and the potential to stop light altogether is limited in practice by the lifetime of light, which is dominated by intrinsic and extrinsic losses. \autocite{Baba2008}\\

%The major component of the higher-order dispersion is the group-velocity dispersion (GVD), given by $d(v_g^{-1})/d\omega = d^2k/d\omega^2$. It usually becomes extremely large near the band edge; a typical GVD constant is of the order of \SI{100}{\ps\per\nm\per\mm}, which is $10^6$ times larger than that of single-mode silica fibres. Because of this, dispersion-compensated and zero-dispersion slow light are very important. \autocite{Baba2008}
%\end{frame}

%\begin{frame}
%Stable measurements are very difficult to perform for these cav- ities [9,29] because they require a stable setup with excellent wavelength resolution. \autocite{Tanabe2007}\\
%Since a high-Q system has a long photon life-time, a time-domain measurement is potentially more accurate than a spectrum measurement. \autocite{Tanabe2007} (Fig. 4)\\
%We expect time-domain measurement to become an indispensable tool for the characterization of future ultrahigh-Q PhC nanocavities. \autocite{Tanabe2007}\\
%It is particularly important with an ultrahigh-Q cavity system to minimize any temperature fluctuation during the pulse delay measurement. \autocite{Tanabe2007}\\
%\end{frame}

%\begin{frame}
%To obtain slow guided-light modes that feature the combination of low group velocity and vanishing GVD parameter. \autocite{Rawal2009}\\
%Operation below the silica light-line, because the modes which lie above the light-line are intrinsically lossy (i.e. leaky) in the vertical direction. \autocite{Rawal2009}\\
%A flat section of dispersion curve should be obtained i.e. the slope of the dispersion curve should not only be small but it should also be close to constant for a given range. \autocite{Rawal2009}\\
%\end{frame}